\section{Conclusions}
\label{sec:conc}

The determination of image provenance is a difficult task to solve. The complexity increases significantly when considering an end-to-end, fully-automatic provenance pipeline that performs at scale. This is the first work, to our knowledge, to have proposed such a technique, and we consider these experiments an important demonstration of the feasibility of large-scale provenance systems.

Our pipeline included an image indexing scheme that utilizes a novel iterative filtering and distributed interest point selection to provide results that outperform the current state-of-the-art found in \cite{pinto2017filtering}. %, while simultaneously improving index computation speeds significantly.
We also proposed methods for provenance graph building that improve upon the methods of previous work in the field, and  provided a novel clustering algorithm for further graph improvement.

To analyze these methods, we utilized the NIST Nimble Challenge~\cite{Nimble_2017} and the multiple-parent phylogeny Professional dataset~\cite{Oliveira_2016} to generate detailed performance results. Beyond utilizing these datasets, we committed to real-world provenance analysis by building our own dataset from Reddit~\cite{reddit2017photoshopbattles}, consisting of unique manipulation scenarios that were generated in an unconstrained environment. This is the first work of its kind to analyze fully in-the-wild provenance cases. 

Upon scrutinizing the results from the three differently sourced datasets, we observed that the proposed approaches perform decently well in connecting the correct set of images (with reported vertex overlaps of nearly 0.8), but still struggle when inferring edge directions --- a result that highlights the difficulty of this problem. Directed  edges are dependent on whether the transformations are reversible or can be inferred from pixel information. In this attempt to perform provenance analysis, we found that although image content is the most reliable source of information connecting related images, other external information may be required to supplement the knowledge obtained from pixels. This external information can be obtained from file metadata, object detectors and compression factors, whenever available.
 
%One major difficulty yet to be overcome in the case of provenance filtering is the detection and retrieval of small or non-salient alien objects within a composite image. While Iterative Filtering has been shown to help improve performance in these cases, we still find that the majority of images not correctly retrieved correspond to these small alien objects. We plan to explore the benefits of a new interest point selection and description algorithm, possibly utilizing deep local features and attention models, to replace SURF as the primary image characterization mechanism for filtering.

%Additionally, this work does not currently utilize previous work found in the Blind Digital Image Forensics (BDIF) field. Significant improvements in region localization, provenance edge calculation, and even edge directionality estimation could be realized by using systems already created in the BDIF field. We plan to explore the benefits of integrating splicing and copy-move detectors, along with noise \cite{}, Photo Response Non-Uniformity (PRNU) \cite{fridrich2009digital}, and Color Filter Array (CFA) models into our pipeline for detecting image inconsistencies and building higher accuracy dissimilarity matrices.

Work in this field is far from complete. The problem of unconstrained, fully-automatic image provenance analysis is not solved.  For instance, this work does not currently utilize previous work found in the Blind Digital Image Forensics (BDIF) field. Significant improvements in region localization, provenance edge calculation, and even edge direction estimation could be performed by using systems already created in the BDIF field. We plan to explore the benefits of integrating splicing and copy-move detectors, along with Photo Response Non-Uniformity (PRNU), %\cite{fridrich2009digital},
and Color Filter Array (CFA) models into our pipeline for detecting image inconsistencies and building higher accuracy dissimilarity matrices.

While this work is a significant first step, we hope to spur others on to further investigate fully-automatic image forensics systems. As the landscapes of social and journalistic media change, so must the field of image forensics adapt with them. News stories, cultural trends, and social sentiments flow at a fast pace, often fueled by unchecked viral images and videos. There is a pressing need to find new solutions and approaches to combat forgery and misinformation. Further, the dual-use nature of such systems makes them useful for other applications, such as cultural analytics, where image provenance can be a primary object of study. We encourage researchers to think broadly when it comes to image provenance analysis.
