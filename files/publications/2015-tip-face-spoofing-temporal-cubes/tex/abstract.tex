\minor{Despite important recent advances, the vulnerability of biometric systems to spoofing attacks is still an open problem. Spoof attacks occur when impostor users present synthetic biometric samples of a valid user to the biometric system seeking to deceive it. Considering the case of face biometrics, a spoofing attack consists in presenting a fake sample (e.g., photograph, digital video or even a 3D mask) to the acquisition sensor with the facial information of a valid user. In this paper, we introduce a low-cost and software-based method for detecting spoofing attempts in face recognition systems. Our hypothesis is that during acquisition there will be inevitable artifacts left behind in the recaptured biometric samples allowing us to create a discriminative signature of the video generated by the biometric sensor. To characterize these artifacts, we extract time-spectral feature descriptors from the video, which can be understood as a low-level feature descriptor that gathers temporal and spectral information across the biometric sample and use the visual codebook concept to find mid-level feature descriptors computed from the low-level ones. Such descriptors are more robust for detecting several kinds of attacks than low-level ones. Experimental results show the effectiveness of the proposed method for detecting different types of attacks in a variety of scenarios and datasets including photos, videos and 3D masks.}
