\section{Related Work}\label{sec:relatedwork}
\minor{The existing techniques for detecting spoofing on face recognition methods can be roughly categorized into four groups: user behavior modeling, user cooperation, methods that require additional hardware and methods based on data-driven characterization. The first aims at modeling the user behavior with respect to the acquisition sensor (e.g., eye blinking or small head and face movements) to decide whether a captured biometric sample is synthetic. Methods based on user cooperation can be used to detect spoofing by means of challenge questions or by asking the user to perform specific movements, which adds extra time and removes the naturalness inherent to facial recognition systems. Techniques that require extra hardware (e.g., infrared cameras or motion and depth sensors) use the additional information generated by these sensors to detect possible clues of an attempted attack. Finally, methods based on data-driven characterization exploit only the data \redmark{captured by the acquisition sensor} looking for evidence and artifacts that may reveal an attempted attack.}

\minor{In~\cite{Pan:ICCV:2007,Xu:ICIP:2008,Li:ICMLC:2008}, the authors proposed a solution for detecting photo-based attacks by eye blinking modeling under the assumption that an attempted attack with photographs differs from valid access by the absence of movements. Bao et al.~\cite{Bao:ICIASP:2009} and Kollreider et al.~\cite{Kollreider:IVC:2009} proposed a method based on the analysis of the characteristics of the optical flow field generated for living faces and photo-based attacks. As a living face is a 3D object and a {photograph is a planar object}, these methods analyze sequential images to detect facial movements, facial expressions or parts of the face such as mouth and eye. Pan et al.~\cite{Pan:TS:2011} extended upon~\cite{Pan:ICCV:2007} including contextual information of the scene (clues outside of the face) and eye blinking (clues inside the face region).}

\minor{Methods that use extra hardware have also been considered in the literature. Sun et al.~\cite{Sun:CAIP:2011} proposed a solution based on thermal IR spectrum modeling the face in the cross-modality of thermal IR and visible light spectrum by canonical correlation analysis. Recently, Erdogmus et al.~\cite{Erdogmus:BTAS:2013} evaluated the behavior of a face biometric system protected with anti-spoofing solutions~\cite{ Chingovska:ICB:2013,Maatta:IJCB:2011} and the \redmark{Microsoft's Kinect} under attempted attacks performed with static 3D masks. Although these approaches were successful, techniques requiring extra hardware devices have the disadvantage of not being possible to implement in computational devices that do not support them, such as smartphones and tablets.}

Turning our attention to the data-driven characterization methods, we can identify three different approaches explored in the literature: methods based on frequency analysis~\cite{Li:BTHI:2004, Pinto:SIBGRAPI:2012, Lee:ICASSP:2013}, texture analysis~\cite{Tan:ECCV:2010, Peixoto:ICIP:2011, Maatta:IJCB:2011, Schwartz:IJCB:2011, Maatta:IET:2012, Kim:ICB:2012, Komulainen:ACCV:2012}, and the ones based on motion and clues of the scene analysis~\cite{Tronci:IJCB:2011, Chingovska:BIOSEG:2012, Yan:ICARCV:2012, Zhang:ICB:2012, Anjos:IJCB:2011}. \allan{We shall briefly review these approaches in the next sections. For further reading on the problem, we recommend Galbally~et~al.'s survey~\cite{Galbally:IEEEAcess:2014} and Marcel~et~al.'s handbook~\cite{Marcel:HBA:2014}.}


%\todo{aqui fala que hah tres abordagens para data-driven, mas quando comeca a descricao dos metodos nos proximos paragrafos, eles nao estao agrupados nessas tres categorias, aih esse paragrafo ficou solto aqui. Seria interessante relacionar os metodos descritos a seguir com essas tres categorias.}

\subsection{Frequency-based approaches}
Li et al.~\cite{Li:BTHI:2004} explored the fact that faces in photographs are smaller than the real ones and that the expressions and poses of the faces in the photographs are invariant to devise a method for detecting photo-based attempted attacks. 

Pinto et al.~\cite{Pinto:SIBGRAPI:2012} proposed a method for detecting attacks performed with videos using visual rhythm analysis. According to the authors, in a video-based spoofing attack, a noise signature is added to the biometric samples during the recapture of the videos of attacks. The authors isolated the noise signal using a low-pass filter  and used the visual rhythm technique to capture the temporal information of the video.

Lee et al.~\cite{Lee:ICASSP:2013} proposed a method based on the frequency entropy of image sequences. The authors used a face verification algorithm to find the face region, normalized the RGB channels using \minor{$z$-score technique}, and applied the independent components analysis (ICA) method to remove cross-channel noise caused by interference from the environment. Finally, the authors calculated the power spectrum and analyzed the entropy of the channels individually. Based on a threshold, the authors decide whether a biometric sample is synthetic or real.

\subsection{Texture-based approaches}
Tan et al.~\cite{Tan:ECCV:2010} proposed a solution for detecting attacks with printed photographs motivated by the difference of the surface roughness of an attempted attack and a real face. The authors estimate the luminance and reflectance of the image under analysis and classify them using Sparse Low Rank Bilinear Logistic Regression methods. Their work was further extended by Peixoto et al.~\cite{Peixoto:ICIP:2011} by incorporating measures for different illumination conditions.

M\"{a}\"{a}tt\"{a} et al.~\cite{Maatta:IJCB:2011} explored micro textures for spoofing detection through the Local Binary Pattern (LBP). To find a holistic representation of the face, able to reveal an attempted attack, Schwartz et al.~\cite{Schwartz:IJCB:2011} proposed a method that extracts different \redmark{information} from images (e.g., color, texture and shape of the face). Results of both techniques were reported in the Competition on Counter Measures to 2D Facial Spoofing Attacks~\cite{Chakka:IJCB:2011}, with an HTER of $0.00\%$ and $0.63\%$, respectively, upon the Print Attack Database~\cite{Anjos:IJCB:2011}.

Chingovska et al.~\cite{Chingovska:BIOSEG:2012} investigated the use of different variations of the LBP operator used in~\cite{Maatta:IJCB:2011}, such as LBP$^{u2}_{3 \times 3}$, tLBP , dLBP and mLBP. The histograms generated from these descriptors were classified using $\chi^{2}$ histogram comparison, Linear Discriminant Analysis and Support Vector Machine. 

Face spoofing attacks performed with static masks have also been considered in the literature. Erdogmus et al.~\cite{Erdogmus:BIOSIG:2013} explored a database with six types of attacks using facial information of four subjects. To detect attempted attacks, the authors used two algorithms based on Gabor wavelet~\cite{Zhang:ICCV:2005, Wiskott:TPAMI:1997} with a Gabor-phase based similarity measure~\cite{Gunther:ICANN:2012}. 

%Kose et al.~\cite{Kose:ICASSP:2013} demonstrated that a face verification system is vulnerable to attacks and, in~\cite{Kose:FG:2013}, Kose et al. evaluated the anti-spoofing method proposed in~\cite{Maatta:IJCB:2011} which was originally proposed to detect photo-based spoofing attacks.

Similarly to Tan et al.~\cite{Tan:ECCV:2010}, Kose et al.~\cite{Kose:DSP:2013} evaluated a solution based on reflectance to detect attacks performed with masks. To decompose the images into components of illumination and reflectance, the Variational Retinex~\cite{Almoussa:UCLA:2009} algorithm was applied.

Pereira et al.~\cite{Pereira:ICB:2013} proposed a score-level fusion strategy for detecting various types of attacks. The authors trained classifiers using different databases and used the $Q$ statistic to evaluate the dependency between classifiers. In a follow-up work, Pereira et al.~\cite{Pereira:JIVP:2014} proposed an anti-spoofing solution based on the dynamic texture, a spatio-temporal version of the original LBP. Results showed that LBP-based dynamic texture description has a higher effectiveness than the original LBP, which reinforces the idea that temporal information is of prime importance to detect spoofing \redmark{attacks}. 

\subsection{Motion-based approaches}

\allan{Tronci et al.~\cite{Tronci:IJCB:2011} explored the motion information and clues that are extracted from the scene by combining two types of processes, referred to as static and video-based analysis. The static analysis consists in combining different visual features such as color, edge, and  Gabor textures, whereas the video-based analysis combines simple motion-related measures such as eye blink, mouth movement, and facial expression change.}

\allan{Anjos et al.~\cite{Anjos:IJCB:2011} proposed a method for detecting photo-based attacks assuming a stationary facial recognition system. According to the authors, the intensity of the relative motion between the face region and the background can be used as a clue to distinguish valid access of attempted attacks, since that motion variations between face and background regions exhibit greater correlation in the case of attempted attacks. %The authors validated the method through the Print-Attack Database~\cite{Anjos:IJCB:2011}.
}

%\todo{senti falta de um contraste do metodo sendo proposto com esses metodos descritos (ficou uma listagem dos metodos existentes sem contextualizar o metodo proposto) - talvez enfatizar que, diferente dos outros, o nosso eh baseado em tempo e dicionario visual e tenta capturar mais tipos de ruidos que os metodos existentes. Seria o caso de avaliar se vale a pena adicionar um paragrafo aqui no final sobre isso.}

\allan{In contrast with the methods described in this section, we present in this work a new anti-spoofing solution based on a temporal characterization of the frequency components from the noise signal extracted from videos. Furthermore, to the best of our knowledge, this was the first attempt of dealing with visual codebooks to find a mid-level representation useful for face spoofing attack detection.} 



