\pdfoutput=1 % required by arXiv
% twocolumn - singlespace
\documentclass[journal]{IEEEtran}
\newif\iffinal
\finaltrue

%% onecolumn - doublesplaced - for review
%\documentclass[11pt,onecolumn,draftcls,journal]{IEEEtran}
%\usepackage[figuresonly]{endfloat} %% required endfloat package version 2.5 for figuresonly option to work
%\newif\iffinal
%\finalfalse



%\usepackage{ifpdf}
% \ifpdf
%   % pdf code
% \else
%   % dvi code
% \fi
%\usepackage{cite}

\ifCLASSINFOpdf
  \usepackage[pdftex]{graphicx}
  % declare the path(s) where your graphic files are
  \graphicspath{{./figs/}{./jpeg/}}
  \DeclareGraphicsExtensions{.pdf,.png}
\else
  % \usepackage[dvips]{graphicx}
  % \graphicspath{{../eps/}}
  % \DeclareGraphicsExtensions{.eps}
\fi
\usepackage[utf8]{inputenc}
\usepackage{flushend}
\usepackage{enumerate}

%\usepackage[cmex10]{amsmath}
%\usepackage{algorithmic}
%\usepackage{array}
\ifCLASSOPTIONcompsoc
  \usepackage[caption=false,font=normalsize,labelfont=sf,textfont=sf]{subfig}
\else
  \usepackage[caption=false,font=footnotesize]{subfig}
\fi
%\usepackage{fixltx2e}
%\usepackage{stfloats}
%\fnbelowfloat
%\usepackage{dblfloatfix}
%\ifCLASSOPTIONcaptionsoff
%  \usepackage[nomarkers]{endfloat}
% \let\MYoriglatexcaption\caption
% \renewcommand{\caption}[2][\relax]{\MYoriglatexcaption[#2]{#2}}
%\fi
\usepackage{url}
\usepackage{hyperref}
\usepackage{balance}

\usepackage{multirow} % - Giovani - 27/06/2014
\usepackage{marvosym} % - Giovani - 03/07/2014

%%% for ToDo command - colored edition
\usepackage{color,colortbl}
\usepackage{hhline}
\definecolor{DarkGray}{rgb}{0.25,0.25,0.25}
\definecolor{LightGray}{rgb}{0.75,0.75,0.75}

\usepackage[table]{xcolor}
\newcommand{\TODO}[1]{\textcolor{black}{[\textbf{TODO: #1}]}}
\newcommand{\ToDo}[1]{\textcolor{black}{[\textbf{ToDo: #1}]}}
\newcommand\blue[1]{{\color{black}#1}}
\newcommand\red[1]{{\color{black}{#1}}}

\newcommand\rva[1]{{\color{black}{#1}}}
\newcommand\rvg[1]{{\color{black}{#1}}}
\newcommand\rrv[1]{{\color{black}#1}}
\newcommand\question[2]{{\color{red}{#1}}{\color{black}{ [#2]}}}

\newcommand{\dm}[1]{\textcolor{green}{{\bf{\textsf{\underline{Menotti}: #1\\}}}}}
\newcommand{\DM}[1]{\textcolor{green}{#1\\}}
%%%

\hyphenation{menottid spoofnet}


\begin{document}
\sloppy

% A Unified Framework to Learn Deep Representations for Biometric Spoofing Modalities
% \title{A Unified Framework to Learn Deep Representations for Iris, Face, and Fingerprint Spoofing Attacks}
\title{Deep Representations for Iris, Face, and Fingerprint Spoofing Detection}
% through Convolutional Neural Networks 

\author{David~Menotti\textsuperscript,~\IEEEmembership{Member,~IEEE,}
        Giovani~Chiachia\textsuperscript,
        Allan~Pinto,~\IEEEmembership{Student Member,~IEEE,}
        William~Robson~Schwartz,~\IEEEmembership{Member,~IEEE,}
        Helio~Pedrini,~\IEEEmembership{Member,~IEEE,}\\
        Alexandre~Xavier~Falc\~{a}o,~\IEEEmembership{Member,~IEEE,}
        and~Anderson~Rocha,~\IEEEmembership{Member,~IEEE}% <-this % stops a space
\thanks{D. Menotti, G. Chiachia, A. Pinto, H. Pedrini, A. X. Falc\~{a}o and A. Rocha are with the Institute of Computing (IC), University of Campinas, Campinas (UNICAMP), SP, 13083-852, Brazil. email: menottid@gmail.com, \{chiachia,allan.pinto,helio.pedrini,afalcao,anderson.rocha\}@ic.unicamp.br.}% <-this % stops a space
\thanks{W. R. Schwartz is with the Computer Science Department (DCC), Federal University of Minas Gerais (UFMG), Belo Horizonte, MG, 31270-010, Brazil. email: william@dcc.ufmg.br.}% <-this % stops a space
\thanks{D. Menotti is also with Computing Department (DECOM), Federal University of Ouro Preto (UFOP), Ouro Preto, MG, 35400-000, Brazil (he has spent his sabbatical year (2013-2014) at IC-UNICAMP). email: menotti@iceb.ufop.br.}% <-this % stops a space
\thanks{\textit{(David Menotti and Giovani Chiachia contributed equally to this work.)}}%
}

% The paper headers
\markboth{IEEE Transactions on Information Forensics and Security,~Vol.~XX, No.~XX, April~XXXX}%
{Menotti \MakeLowercase{\textit{et al.}}: Deep Representations for Iris, Face, and Fingerprint Spoofing Detection}
% You can use \ifCLASSOPTIONpeerreview for conditional compilation here if
% you desire.

% If you want to put a publisher's ID mark on the page you can do it like this:
%\IEEEpubid{0000--0000/00\$00.00~\copyright~2012 IEEE}
% Remember, if you use this you must call \IEEEpubidadjcol in the second column for its text to clear the IEEEpubid mark.


% use for special paper notices
%\IEEEspecialpapernotice{(Invited Paper)}

% make the title area
\maketitle


\begin{abstract}
Biometrics systems have significantly improved person
identification and authentication, playing an important role in
personal, national, and global security. However, these systems
might be deceived (or ``spoofed'') and, despite the recent advances
in spoofing detection, current solutions often rely on domain
knowledge, specific biometric reading systems, and attack types. We
assume a very limited knowledge about biometric spoofing at the
sensor to derive outstanding spoofing detection systems for iris,
face, and fingerprint modalities based on two deep learning
approaches.
The first approach consists of learning suitable convolutional 
network architectures for each domain, while the second approach
focuses on learning the weights of the network via back-propagation. We
consider nine biometric spoofing benchmarks --- each one containing
real and fake samples of a given biometric modality and attack type
--- and learn deep representations for each benchmark by combining and
contrasting the two learning approaches. This strategy not only
provides better comprehension of how these approaches interplay, but
also creates systems that exceed the best known results in eight out
of the nine benchmarks. The results strongly indicate that spoofing
detection systems based on convolutional networks can be robust to
attacks already known and possibly adapted, with little effort, to
image-based attacks that are yet to come.

% Biometrics systems have drastically improved person identification
% and/or authentication, playing an important role in personal,
% national, and global security. However, such systems might be
% deceived or ``spoofed'' and, despite the advances in
% spoofing detection to each biometric modality, there is still a lack
% of a clearly unified approach. Aiming at filling this gap, we
% propose a unified framework to learn deep representations for three
% different modalities of spoofing biometrics (i.e., face, iris, and
% fingerprint). The representations are learned directly from the
% data with an optimization procedure that randomly searches for the best
% convolutional neural network from a family of networks defined in a hyperparameter search
% space. Atop these representations, we couple a linear Support Vector Machine
% classifiers for the final
% decision making. Instead of learning thousands and/or millions of
% network weights/parameters based either on back-propagation-like or unsupervised
% feature learning techniques, our networks use random filter weights initialized in a 
% convenient way, which allows us to quickly probe thousands of network configurations.
% Experiments on nine publicly available benchmarks of these 
% three modalities show that our framework achieves either very 
% competitive or state-of-the-art results for all problems and modalities.

% Biometrics systems have drastically improved person identification
% and/or authentication, playing an important role in personal,
% national, and global security. However, such systems might be
% deceived or ``spoofed'' and, despite the advances in
% spoofing detection to each biometric modality, there is still a lack
% of a clearly unified approach. Aiming at filling this gap, we
% propose a unified framework to learn deep representations for three
% different modalities of spoofing biometrics (i.e., face, iris, and
% fingerprint). The representations are learned directly from the
% data, as statistical models of the distributions of the
% real and fake classes, using convolutional neural networks. On the
% top of these representations, we couple a Support Vector Machine
% classifier with linear kernel and hard margins for the final
% decision making. Instead of learning thousands and/or millions of
% network weights/parameters, based either on back-propagation-like or unsupervised
% learning techniques, we focus on the random optimization of the
% network architecture with zero-mean and unit-norm uniform random
% weights. Experiments on nine publicly available benchmarks of these 
% three modalities show that our framework achieves either very 
% competitive or state-of-the-art results for all problems and modalities.
\end{abstract}

% Note that keywords are not normally used for peerreview papers.
\begin{IEEEkeywords}
Deep Learning, Convolutional Networks, Hyperparameter Architecture Optimization, Filter Weights Learning, Back-propagation, Spoofing Detection.
\end{IEEEkeywords}

% For peer review papers, you can put extra information on the cover
% page as needed:
% \ifCLASSOPTIONpeerreview
% \begin{center} \bfseries EDICS Category: 3-BBND \end{center}
% \fi
%
% For peerreview papers, this IEEEtran command inserts a page break and
% creates the second title. It will be ignored for other modes.
\IEEEpeerreviewmaketitle


% Please do not TOUCH the list or names below. Anderson is reviewing
\input{introduction} % rereview by Menotti on 1h30
\section{Related Work}\label{sec:relatedwork}
\minor{The existing techniques for detecting spoofing on face recognition methods can be roughly categorized into four groups: user behavior modeling, user cooperation, methods that require additional hardware and methods based on data-driven characterization. The first aims at modeling the user behavior with respect to the acquisition sensor (e.g., eye blinking or small head and face movements) to decide whether a captured biometric sample is synthetic. Methods based on user cooperation can be used to detect spoofing by means of challenge questions or by asking the user to perform specific movements, which adds extra time and removes the naturalness inherent to facial recognition systems. Techniques that require extra hardware (e.g., infrared cameras or motion and depth sensors) use the additional information generated by these sensors to detect possible clues of an attempted attack. Finally, methods based on data-driven characterization exploit only the data \redmark{captured by the acquisition sensor} looking for evidence and artifacts that may reveal an attempted attack.}

\minor{In~\cite{Pan:ICCV:2007,Xu:ICIP:2008,Li:ICMLC:2008}, the authors proposed a solution for detecting photo-based attacks by eye blinking modeling under the assumption that an attempted attack with photographs differs from valid access by the absence of movements. Bao et al.~\cite{Bao:ICIASP:2009} and Kollreider et al.~\cite{Kollreider:IVC:2009} proposed a method based on the analysis of the characteristics of the optical flow field generated for living faces and photo-based attacks. As a living face is a 3D object and a {photograph is a planar object}, these methods analyze sequential images to detect facial movements, facial expressions or parts of the face such as mouth and eye. Pan et al.~\cite{Pan:TS:2011} extended upon~\cite{Pan:ICCV:2007} including contextual information of the scene (clues outside of the face) and eye blinking (clues inside the face region).}

\minor{Methods that use extra hardware have also been considered in the literature. Sun et al.~\cite{Sun:CAIP:2011} proposed a solution based on thermal IR spectrum modeling the face in the cross-modality of thermal IR and visible light spectrum by canonical correlation analysis. Recently, Erdogmus et al.~\cite{Erdogmus:BTAS:2013} evaluated the behavior of a face biometric system protected with anti-spoofing solutions~\cite{ Chingovska:ICB:2013,Maatta:IJCB:2011} and the \redmark{Microsoft's Kinect} under attempted attacks performed with static 3D masks. Although these approaches were successful, techniques requiring extra hardware devices have the disadvantage of not being possible to implement in computational devices that do not support them, such as smartphones and tablets.}

Turning our attention to the data-driven characterization methods, we can identify three different approaches explored in the literature: methods based on frequency analysis~\cite{Li:BTHI:2004, Pinto:SIBGRAPI:2012, Lee:ICASSP:2013}, texture analysis~\cite{Tan:ECCV:2010, Peixoto:ICIP:2011, Maatta:IJCB:2011, Schwartz:IJCB:2011, Maatta:IET:2012, Kim:ICB:2012, Komulainen:ACCV:2012}, and the ones based on motion and clues of the scene analysis~\cite{Tronci:IJCB:2011, Chingovska:BIOSEG:2012, Yan:ICARCV:2012, Zhang:ICB:2012, Anjos:IJCB:2011}. \allan{We shall briefly review these approaches in the next sections. For further reading on the problem, we recommend Galbally~et~al.'s survey~\cite{Galbally:IEEEAcess:2014} and Marcel~et~al.'s handbook~\cite{Marcel:HBA:2014}.}


%\todo{aqui fala que hah tres abordagens para data-driven, mas quando comeca a descricao dos metodos nos proximos paragrafos, eles nao estao agrupados nessas tres categorias, aih esse paragrafo ficou solto aqui. Seria interessante relacionar os metodos descritos a seguir com essas tres categorias.}

\subsection{Frequency-based approaches}
Li et al.~\cite{Li:BTHI:2004} explored the fact that faces in photographs are smaller than the real ones and that the expressions and poses of the faces in the photographs are invariant to devise a method for detecting photo-based attempted attacks. 

Pinto et al.~\cite{Pinto:SIBGRAPI:2012} proposed a method for detecting attacks performed with videos using visual rhythm analysis. According to the authors, in a video-based spoofing attack, a noise signature is added to the biometric samples during the recapture of the videos of attacks. The authors isolated the noise signal using a low-pass filter  and used the visual rhythm technique to capture the temporal information of the video.

Lee et al.~\cite{Lee:ICASSP:2013} proposed a method based on the frequency entropy of image sequences. The authors used a face verification algorithm to find the face region, normalized the RGB channels using \minor{$z$-score technique}, and applied the independent components analysis (ICA) method to remove cross-channel noise caused by interference from the environment. Finally, the authors calculated the power spectrum and analyzed the entropy of the channels individually. Based on a threshold, the authors decide whether a biometric sample is synthetic or real.

\subsection{Texture-based approaches}
Tan et al.~\cite{Tan:ECCV:2010} proposed a solution for detecting attacks with printed photographs motivated by the difference of the surface roughness of an attempted attack and a real face. The authors estimate the luminance and reflectance of the image under analysis and classify them using Sparse Low Rank Bilinear Logistic Regression methods. Their work was further extended by Peixoto et al.~\cite{Peixoto:ICIP:2011} by incorporating measures for different illumination conditions.

M\"{a}\"{a}tt\"{a} et al.~\cite{Maatta:IJCB:2011} explored micro textures for spoofing detection through the Local Binary Pattern (LBP). To find a holistic representation of the face, able to reveal an attempted attack, Schwartz et al.~\cite{Schwartz:IJCB:2011} proposed a method that extracts different \redmark{information} from images (e.g., color, texture and shape of the face). Results of both techniques were reported in the Competition on Counter Measures to 2D Facial Spoofing Attacks~\cite{Chakka:IJCB:2011}, with an HTER of $0.00\%$ and $0.63\%$, respectively, upon the Print Attack Database~\cite{Anjos:IJCB:2011}.

Chingovska et al.~\cite{Chingovska:BIOSEG:2012} investigated the use of different variations of the LBP operator used in~\cite{Maatta:IJCB:2011}, such as LBP$^{u2}_{3 \times 3}$, tLBP , dLBP and mLBP. The histograms generated from these descriptors were classified using $\chi^{2}$ histogram comparison, Linear Discriminant Analysis and Support Vector Machine. 

Face spoofing attacks performed with static masks have also been considered in the literature. Erdogmus et al.~\cite{Erdogmus:BIOSIG:2013} explored a database with six types of attacks using facial information of four subjects. To detect attempted attacks, the authors used two algorithms based on Gabor wavelet~\cite{Zhang:ICCV:2005, Wiskott:TPAMI:1997} with a Gabor-phase based similarity measure~\cite{Gunther:ICANN:2012}. 

%Kose et al.~\cite{Kose:ICASSP:2013} demonstrated that a face verification system is vulnerable to attacks and, in~\cite{Kose:FG:2013}, Kose et al. evaluated the anti-spoofing method proposed in~\cite{Maatta:IJCB:2011} which was originally proposed to detect photo-based spoofing attacks.

Similarly to Tan et al.~\cite{Tan:ECCV:2010}, Kose et al.~\cite{Kose:DSP:2013} evaluated a solution based on reflectance to detect attacks performed with masks. To decompose the images into components of illumination and reflectance, the Variational Retinex~\cite{Almoussa:UCLA:2009} algorithm was applied.

Pereira et al.~\cite{Pereira:ICB:2013} proposed a score-level fusion strategy for detecting various types of attacks. The authors trained classifiers using different databases and used the $Q$ statistic to evaluate the dependency between classifiers. In a follow-up work, Pereira et al.~\cite{Pereira:JIVP:2014} proposed an anti-spoofing solution based on the dynamic texture, a spatio-temporal version of the original LBP. Results showed that LBP-based dynamic texture description has a higher effectiveness than the original LBP, which reinforces the idea that temporal information is of prime importance to detect spoofing \redmark{attacks}. 

\subsection{Motion-based approaches}

\allan{Tronci et al.~\cite{Tronci:IJCB:2011} explored the motion information and clues that are extracted from the scene by combining two types of processes, referred to as static and video-based analysis. The static analysis consists in combining different visual features such as color, edge, and  Gabor textures, whereas the video-based analysis combines simple motion-related measures such as eye blink, mouth movement, and facial expression change.}

\allan{Anjos et al.~\cite{Anjos:IJCB:2011} proposed a method for detecting photo-based attacks assuming a stationary facial recognition system. According to the authors, the intensity of the relative motion between the face region and the background can be used as a clue to distinguish valid access of attempted attacks, since that motion variations between face and background regions exhibit greater correlation in the case of attempted attacks. %The authors validated the method through the Print-Attack Database~\cite{Anjos:IJCB:2011}.
}

%\todo{senti falta de um contraste do metodo sendo proposto com esses metodos descritos (ficou uma listagem dos metodos existentes sem contextualizar o metodo proposto) - talvez enfatizar que, diferente dos outros, o nosso eh baseado em tempo e dicionario visual e tenta capturar mais tipos de ruidos que os metodos existentes. Seria o caso de avaliar se vale a pena adicionar um paragrafo aqui no final sobre isso.}

\allan{In contrast with the methods described in this section, we present in this work a new anti-spoofing solution based on a temporal characterization of the frequency components from the noise signal extracted from videos. Furthermore, to the best of our knowledge, this was the first attempt of dealing with visual codebooks to find a mid-level representation useful for face spoofing attack detection.} 



 % order background first?
%!TEX root = 2014-TIFS-dl-spoofing.tex
\section{Benchmarks}
\label{sec:databases}

In this section, we describe the benchmarks (datasets) that we consider in this work. All them are publicly available upon request and suitable for evaluating countermeasure methods to iris, face and fingerprint spoofing attacks. Table~\ref{tab:databases} shows the major features of each one and in the following we describe their details.


\begin{table*}[tb!]
\begin{center}
\caption{Main features of the benchmarks considered herein.}
\label{tab:databases}
%\tiny  \scriptsize \footnotesize \small \normalsize     
\iffinal
%\small
\else
\tiny
\fi
\begin{tabular}{clccrrrcrrrcrrr}
\hline
\multirow{2}{*}{Modality}
& \multirow{2}{*}{Benchmark/Dataset}
                                            & \multirow{2}{*}{Color}
                                                    &      Dimension
                                                                         &\multicolumn{3}{c}{\# Training}
                                                                                               && \multicolumn{3}{c}{\# Testing} 
                                                                                                                     && \multicolumn{3}{c}{\# Development} \\
\cline{5-7}\cline{9-11}\cline{13-15}
&                                            &       & $cols \times rows$ & Live & Fake & Total && Live & Fake & Total && Live & Fake & Total \\
\hline
\hline
\multirow{3}{*}{Iris}
&Warsaw~\cite{Czajka:MMAR:2013}              & No    & $640 \times  480$ &  228 &  203 &   431 &&  624 &  612 &  1236 \\
&Biosec~\cite{Ruiz-Albacete:BIOID:2008}      & No    & $640 \times  480$ &  200 &  200 &   400 &&  600 &  600 &  1200 \\
&MobBIOfake~\cite{Sequeira:VISAPP:2014:base} & Yes   & $250 \times  200$ &  400 &  400 &   800 &&  400 &  400 &   800 \\
\hline
\multirow{2}{*}{Face}
& Replay-Attack~\cite{Chakka:IJCB:2011}      & Yes   & $320 \times  240$ &  600 & 3000 &  3600 && 4000 &  800 &  4800 &&  600 & 3000 & 3600 \\
& 3dMad~\cite{Chingovska:ICB:2013}           & Yes   & $640 \times  480$ &  350 &  350 &   700 &&  250 &  250 &   500 &&  250 &  250 &  500 \\
\hline
\multirow{4}{*}{Fingerprint} 
&Biometrika~\cite{Ghiani:ICB:2013}           & No    & $312 \times  372$ & 1000 & 1000 &  2000 && 1000 & 1000 &  2000 \\
&CrossMatch~\cite{Ghiani:ICB:2013}           & No    & $800 \times  750$ & 1250 & 1000 &  2250 && 1250 & 1000 &  2250 \\
&Italdata~\cite{Ghiani:ICB:2013}             & No    & $640 \times  480$ & 1000 & 1000 &  2000 && 1200 & 1000 &  2000 \\
&Swipe~\cite{Ghiani:ICB:2013}                & No    & $208 \times 1500$ & 1221 &  979 &  2200 && 1153 & 1000 &  2153 \\
\hline

\end{tabular}
\end{center}
\end{table*}

\subsection{Iris Spoofing Benchmarks}

\subsubsection{Biosec} 
This benchmark was created using iris images from $50$ users of the BioSec~\cite{Ruiz-Albacete:BIOID:2008}.
In total, there are $16$ images for each user ($2$ sessions $\times$ $2$ eyes $\times$ $4$ images), totalizing $800$ valid access images. 
To create spoofing attempts, the original images from Biosec were preprocessed to improve quality and printed using an HP Deskjet 970cxi and an HP LaserJet 4200L printers. 
Finally, the iris images were recaptured with the same iris camera used to capture the original images.

\subsubsection{Warsaw} 
This benchmark contains $1274$ images of $237$ volunteers representing valid accesses and $729$ printout images representing spoofing attempts, which were generated by using two printers: (1) a HP LaserJet 1320 used to produce $314$ fake images with $600$ dpi resolution, and (2) a Lexmark C534DN used to produce $415$ fake images with $1200$ dpi resolution. Both real and fake images were captured by an IrisGuard AD100 biometric device.

\subsubsection{MobBIOfake} 
This benchmark contains live iris images and fake printed iris images captured with the same acquisition sensor, i.e., a mobile phone. To generate fake images, the authors first performed a preprocessing in the original images to enhance the contrast. The preprocessed images were then printed with a professional printer on high quality photographic paper.
%Os autores não forneceram mais dados sobre este dataset.


\subsection{Video-based Face Spoofing Benchmarks}

\subsubsection{Replay-Attack} 
This benchmark contains short video recordings of both valid accesses and video-based attacks of $50$ different subjects. 
To generate valid access videos, each person was recorded in two sessions in a controlled and in an adverse environment with a regular webcam.
Then, spoofing attempts were generated using three techniques:
(1)~\emph{print attack}, which presents to the acquisition sensor hard copies of high-resolution digital photographs printed with a Triumph-Adler DCC 2520 color laser printer; 
(2) \emph{mobile attack}, which presents to the acquisition sensor photos and videos taken with an iPhone using the iPhone screen; 
and (3) \emph{high-definition attack}, in which high resolution photos and videos taken with an iPad are presented to the acquisition sensor using the iPad screen.
%In total, there is $200$ valid access videos and $1000$ spoofing attack videos.

\subsubsection{3DMAD} 
This benchmark consists of real videos and fake videos made with people wearing masks. A total of $17$ different subjects were recorded with a Microsoft Kinect sensor, and videos were collected in three sessions. For each session and each person, five videos of $10$ seconds were captured. The 3D masks were produced by \url{ThatsMyFace.com} using one frontal and two profile images of each subject.
All videos were recorded by the same acquisition sensor.
%In total, there are 85 valid access videos and 85 faces. 
%\ToDo{Allan: how many videos? 85 videos de acesso válido e 85 fakes}


\subsection{Fingerprint Spoofing Benchmarks}

\subsubsection{LivDet2013} 
This dataset contains four sets of real and fake fingerprint readings performed in four acquisition sensors: Biometrika FX2000, Italdata ET10, Crossmatch L Scan Guardian, and Swipe.
For a more realistic scenario, fake samples in Biometrika and Italdata were generated without user cooperation, while fake samples in Crossmatch and Swipe were generated with user cooperation. 
Several materials for creating the artificial fingerprints were used, including gelatin, silicone, latex, among others. 
%Both Biometrika and Italdata sets are composed of $1000$ real images and $1000$ fake images. 
%Already, Crossmatch and Swipe sets are composed of $1250$ real images and $1250$ fake images\footnote{those are the figures reported on the public works, but the real number used in this work is slightly different for the Swipe Dataset}.


%%%%%%%%%%%%%%%%%%%%%%%%%%%%%%%%%%%%%%%%%%%%%%%%%%%%%%%%%%%%%

\subsection{Remark}

% The benchmarks described in this section aim at evaluating the proposed method considering a more realistic scenario, given that heterogeneity of the datasets is a challenge, mainly for algorithms based on machine learning, due to difficulty of finding a generalizable model.

Images found in these benchmarks can be observed in Fig.~\ref{fig:datasets} of Section~\ref{sec:experiments}. 
As we can see, variability exists not only across modalities, but also within modalities. Moreover, it is rather unclear what features might discriminate real from spoofed images, which suggests that the use of a methodology able to use data to its maximum advantage might be a promising idea to tackle such set of problems in a principled way.

% Methods developed to work on some database using specific features might not perform well on other databases of the same problem. 
% This hard scenario hints the use of the proposed framework which aims to learn deep representations from the data of each database such that promising effectiveness is expected to be achieved.




% Even considering each modality alone, we have data from many different sensors, which complicates the problem of detecting an attempted attack, and different modes of spoofing attacks, using different types of equipment and procedures. 

% To have an idea of the difficulty of this problem, consider the images depicted in Fig.~\ref{fig:datasets} (linked to Section~\ref{sec:experiments}).
%\TODO{essa referencia de pagina talvez seja perdida na versao final, melhor deixar apenas o numero da figura}. 
%pp.~\pageref{page:fig:datasets}
% Menotti: concordo que a p�gina n�o � o melhor art�fico, mas a figura est� muito longe, por isso usei "linked to Section~\ref{sec:discussion}
% Observe the high variability even within the same modalities. 
%\question{Experiments in this hard scenario hints at the performance of the proposed framework in a more realistic and operational scenario}%
%{Sentenca confusa, nao estah claro qual eh a conclusao que se queria ter com essa sentenca.}. 
%suggestion
% Methods developed to work on some database using specific features might not perform well on other databases of the same problem. 
% This hard scenario hints the use of the proposed framework which aims to learn deep representations from the data of each database such that promising effectiveness is expected to be achieved.

% In the next section, we describe details of our framework for  spoofing attack detection under different modalities. 
% We shall return to the details in such figure when discussing the experimental results. 

%  
\section{Methodology}
\label{sec:methodology}

In this section, we present the methodology for architecture optimization (AO) and filter optimization (FO) as well as details about how benchmark images are preprocessed, how AO and FO are evaluated across the benchmarks, and how these methods are implemented.

\subsection{Architecture Optimization (AO)}
\label{sec:ao}

Our approach for AO builds upon the work of Pinto et al.~\cite{Pinto:2009} and Bergstra et al.~\cite{Bergstra:2013}, i.e., fundamental, feedforward convolutional operations are stacked by means of hyperparameter optimization, leading to effective yet simple convolutional networks that do not require expensive filter optimization and from which prediction is done by linear support vector machines (SVMs).

Operations in convolutional networks can be viewed as linear and non-linear transformations that, when stacked, extract high level representations of the input. Here we use a well-known set of operations called (i) \emph{convolution} with a bank of filters, (ii) rectified linear \emph{activation}, (iii) spatial \emph{pooling}, and (iv) \emph{local normalization}. Appendix~\ref{sec:convnet_ops} provides a detailed definition of these operations.

We denote as \emph{layer} the combination of these four operations in the order that they appear in the left panel of Fig.~\ref{fig:framework}. Local normalization is optional and its use is governed by an additional ``yes/no'' hyperparameter. In fact, there are other six hyperparameters, each of a particular operation, that have to be defined in order to instantiate a layer. They are presented in the lower part of the left panel in Fig.~\ref{fig:framework} and are in accordance to the definitions of Appendix~\ref{sec:convnet_ops}.

Considering one layer and possible values of each hyperparameter, there are over 3,000 possible layer architectures, and this number grows exponentially with the number of layers, which goes up to three in our case (Fig.~\ref{fig:framework} right panel). In addition, there are network-level hyperparameters, such as the size of the input image, that expand possibilities to a myriad potential architectures.
 
 % and emphasize the need to rapidly evaluate candidate architectures.

 % and that we are interested in learning deep representations extracted with the combination of up to three of such layers (Fig.~\ref{fig:framework} right panel), the importance of hyperparameter optimization in this context becomes clear. 

% Considering one layer and possible values of each hyperparameter, there are over 3,000 possible layer architectures. Given that this number grows exponentially with the number of layers and that we are interested in learning deep representations extracted with the combination of up to three of such layers (Fig.~\ref{fig:framework} right panel), the importance of hyperparameter optimization in this context becomes clear. In addition, there are network-level hyperparameters that expand possibilities even more, like the size of the input image, or the depth of the candidate network.


The overall set of possible hyperparameter values is called \emph{search space}, which in this case is discrete and contains variables that are only meaningful in combination with others. For example, hyperparameters of a given layer are just meaningful if the candidate architecture has actually that number of layers.
In spite of the intrinsic difficulty in optimizing architectures in this space, \emph{random search} has played and important role in problems of this type~\cite{Pinto:2009,Bergstra:2012} and it is the strategy of our choice due to its effectiveness and simplicity.

We can see in Fig.~\ref{fig:framework} that a three-layered network has a total of 25 hyperparameters, seven per layer and four at network level. They are all defined in Appendix~\ref{sec:convnet_ops} with the exception of \emph{input size}, which seeks to determine the best size of the image's greatest axis (rows or columns) while keeping its aspect ratio. Concretely, random search in this paper can be described as follows:

\begin{enumerate}
\item Randomly --- and uniformly, in our case --- sample values from the hyperparameter \emph{search space};
\item Extract features from real and fake training images with the candidate architecture;
\item Evaluate the architecture according to an \emph{optimization objective} based on linear SVM scores;
\item Repeat steps 1--3 until a \emph{termination criterion} is met;
\item Return the best found convolutional architecture.
\\
\end{enumerate}



\begin{figure}
\begin{center}
 \includegraphics[width=1.0\linewidth]{hp-schema.pdf}
 \caption{Schematic diagram for architecture optimization (AO) illustrating how operations are stacked in a layer (left) and how the network is instantiated and evaluated according to possible hyperparameter values (right). Note that a three-layered convolutional network of this type has a total of 25 hyperparameters governing both its architecture and its overall behaviour through a particular instance of stacked operations.}
 \label{fig:framework}
\end{center}
\end{figure}


Even though there are billions of possible networks in the search space (Fig.~\ref{fig:framework}), it is important to remark that not all candidate networks are valid. For example, a large number of candidate architectures (i.e., points in the search space) would produce representations with spatial resolution smaller than one pixel. Hence, they are naturally unfeasible.
Additionally, in order to avoid very large representations, we discard in advance candidate architectures whose intermediate layers produce representations of over 600K elements or whose output representation has over 30K elements.

% Filter optimization is of paramount importance in convolutional networks and 

Filter weights are randomly generated for AO. This strategy has been successfully used in the vision literature~\cite{Pinto:2009,Saxe:2011,Pinto:2011b,Jarrett:2009} and is essential to make AO practical, avoiding the expensive filter optimization (FO) part in the evaluation of candidate architectures. We sample weights from a uniform distribution $U(0,1)$ and normalize the filters to zero mean and unit norm in order to ensure that they are spread over the unit sphere. When coupled with rectified linear activation (Appendix~\ref{sec:convnet_ops}), this sampling enforces sparsity in the network by discarding about $50$\% of the expected filter responses, thereby improving the overall robustness of the feature extraction.

A candidate architecture is evaluated by first extracting deep representations from real and fake images and later training hard-margin linear SVMs ($C$=$10^5$) on these representations.
We observed that the sensitivity of the performance measure was saturating with traditional 10-fold cross validation (CV) in some benchmarks. Therefore, we opted for a different validation strategy. Instead of training on nine folds and validating on one, we train on one fold and validate on nine. Precisely, the \emph{optimization objective} is the mean detection accuracy obtained from this adapted cross-validation scheme, which is maximized during the optimization.

For generating the 10 folds, we took special care in putting all samples of an individual in the same fold to enforce robustness to cross-individual spoofing detection in the optimized architectures.
Moreover, in benchmarks where we have more than one attack type (e.g., Replay-Attack and LivDet2013, see Section~\ref{sec:databases}), we evenly distributed samples of each attack type across all folds in order to enforce that candidate architectures are also robust to different types of attack.

Finally, the \emph{termination criterion} of our AO procedure simply consists of counting the number of valid candidate architectures and stopping the optimization when this number reaches 2,000.


\subsection{Filter Optimization (FO)}
\label{sec:fo}

\begin{figure}
\begin{center}
\includegraphics[width=1.0\linewidth]{cuda-convnet.pdf}
\caption{
Architecture of convolutional network found in the Cuda-convnet library~\cite{Krizhevsky:cuda-convnet:2012} and here used as reference for filter optimization (\emph{cf10-11}, top). Proposed network architecture extending upon~\emph{cf10-11} to better suiting spoofing detection problems (\emph{spoofnet}, bottom). Both architectures are typical examples where domain knowledge has been incorporated for increased performance.
}
\label{fig:cudaconvnet}
\end{center}
\end{figure}

We now turn our attention to a different approach for tackling the problem. Instead of optimizing the architecture, we explore the filter weights and how to learn them for better characterizing real and fake samples. Our approach for FO is at the origins of convolutional networks and consists of learning filter weights via the well-known back-propagation algorithm~\cite{LeCun:1998}. Indeed, due to a refined understanding of the optimization process and the availability of plenty of data and processing power, back-propagation has been the gold standard method in deep networks for computer vision in the last years~\cite{Krizhevsky:2012,Simonyan:2014,Zeiler:2014}.

For optimizing filters, we need to have an already defined architecture. We start optimizing filters with a standard public convolutional network and training procedure. This network is available in the Cuda-convnet library~\cite{Krizhevsky:cuda-convnet:2012} and is currently one of the best performing architectures in CIFAR-10,\footnote{\url{http://www.cs.toronto.edu/~kriz/cifar.html}} a popular computer vision benchmark in which such network achieves 11\% of classification error. Hereinafter, we call this network \emph{cuda-convnet-cifar10-11pct}, or simply \emph{cf10-11}.

Fig.~\ref{fig:cudaconvnet} depicts the architecture of \emph{cf10-11} in the top panel and is a typical example where domain knowledge has been incorporated for increased performance. We can see it as a three-layered network in which the first two layers are convolutional, with operations similar to the operations used in architecture optimization (AO). In the third layer, \emph{cf10-11} has two sublayers of unshared local filtering and a final fully-connected sublayer on top of which softmax regression is performed. A detailed explanation of the operations in \emph{cf10-11} can be found in~\cite{Krizhevsky:cuda-convnet:2012}.

In order to train~\emph{cf10-11} in a given benchmark, we split the training images into four batches observing the same balance of real and fake images. After that, we follow a procedure similar to the original\footnote{\url{https://code.google.com/p/cuda-convnet/wiki/Methodology}.} for training~\emph{cf10-11} in all benchmarks, which can be described as follows:

\begin{enumerate}
\item For 100 epochs, train the network with a learning rate of $10^{-3}$ by considering the first three batches for training and the fourth batch for validation;
\item For another 40 epochs, resume training now considering all four batches for training;
\item Reduce the learning rate by a factor of 10, and train the network for another 10 epochs;
\item Reduce the learning rate by another factor of 10, and train the network for another 10 epochs.
\\
\end{enumerate}

% Once~\emph{cf10-11} is trained in a given benchmark, prediction on test samples is performed by softmax 
% The network learned after this process is the one used to evaluate the benchmarks.

After evaluating filter learning on the \emph{cf10-11} architecture, we also wondered how filter learning could benefit from an optimized architecture incorporating domain-knowledge of the problem. Therefore, extending upon the knowledge obtained with AO as well as with training~\emph{cf10-11} in the benchmarks, we derived a new architecture for spoofing detection that we call~\emph{spoofnet}. Fig.~\ref{fig:cudaconvnet} illustrates this architecture in the bottom panel and has three key differences as compared to~\emph{cf10-11}. First, it has 16 filters in the first layer instead of 64. Second, operations in the second layer are stacked in the same order that we used when optimizing architectures (AO). Third, we removed the two unshared local filtering operations in the third layer, as they seem inappropriate in a problem where object structure is irrelevant.

These three modifications considerably dropped the number of weights in the network and this, in turn, allowed us to increase of size of the input images from $32\times32$ to $128\times128$. This is the fourth and last modification in~\emph{spoofnet}, and we believe that it might enable the network to be more sensitive to subtle local patterns in the images.

In order to train~\emph{spoofnet}, the same procedure used to train~\emph{cf10-11} is considered except for the initial learning rate, which is made $10^{-4}$, and for the number of epochs in each step, which is doubled. These modifications were made because of the decreased learning capacity of the network.
% However, given that~\emph{spoofnet} is much faster than~\emph{cf10-11} to process images, the overall training time was not increased.

% For the extended architecture we proposed here, we have developed a slightly different training methodology that the one used for the Krizhevsky \emph{et al.}'s model.
% Indeed, we increase by two the number of training epochs in the three phases described above adding up 320 epochs, since we reduce the number of filter in the first convolutional layer.
% Moreover, we setup the starting learning rate equal to $10^{-4}$, instead of $10^{-3}$ as done in Krizhevsky's methodology.


% We also propose an extended version of Alex's architecture, which can be seen as a reduced one to the complexity of spoofing problem.
% Indeed, we: 1) use only 16 filters in the first convolutional layer; 2) remove the third layer, the locally-connected layer. 
% Moreover $128 \times 128 \times C$ image pixels are used as input for this extended version.
% These modifications made the evaluation time for the extended network at least twice faster.


% The domain knowledge convolution network chosen to be used here for filter learning via backpropagation is the one used in~\cite{Krizhevsky:2012} which achieves accuracy near $90\%$ on CIFAR-10 dataset. 

% In the following we describe in details these architectures, their training methodologies together with the data preparation for and how data augmentation is employed to boost the learning processing.

% More specifically Krizhevsky \emph{et al.}'s (or simply Alex) architecture~\cite{Krizhevsky:2012} can be seen as composed of four layers. 
% In the first one, convolutional operation with 64 filters of $5\times 5$, followed by a $3\times 3$ max-pooling operation with stride 2, and $9 \times 9$ local response/contrast normalization. 
% In the second layer, convolutional operation with 64 filters of $3 \times 3$, followed by $9\times 9$ local response/contrast normalization and then $3\times 3$ max-pooling operation with stride 3.
% In the third layer, we have two locally-connected operations with unshared weight, using 64 and 32 filters of $3\times 3$, respectively.
% The final layer, which can be seen as the classification one, is composed of a fully connected operation to two neurons attached to softmax operations, which can be interpreted as probabilities, where the backpropation starts the learning task with target classes. For such, an objective function should be defined. In this case a (multinomial) logistic regression objective one.
% This network can also be seen as being composed of 11 layers, if each operation is considered as a layer.

% The input for this network uses $32 \times 32 \times c$ image pixels - the standard size of CIFAR-10 dataset, where $c$ stands for the number of channels in the images of each benchmark. 
% But here we adopt $c=1$ for the benchmarks using grayscale images, and keep $c=3$ when color images are available.

Finally, in order to reduce overfitting, data augmentation is used for training both networks according to the procedure of~\cite{Krizhevsky:2012}. For \emph{cf10-11}, five $24\times24$ image patches are cropped out from the $32\times32$ input images. These patches correspond to the four corners and central region of the original image, and their horizontal reflections are also considered. Therefore, ten training samples are generated from a single image. For \emph{spoofnet}, the procedure is the same except for the fact that input images have $128\times128$ pixels and cropped regions are of $112\times112$ pixels. During prediction, just the central region of the test image is considered.


\subsection{Elementary Preprocessing}
\label{sec:preproc}

A few basic preprocessing operations were executed on face and fingerprint images in order to properly learn representations for these benchmarks. 
This preprocessing led to images with sizes as presented in Table~\ref{tab:databases:experiments} and are described in the next two sections.

\begin{table}[tb!]
\begin{center}
\caption{Input image dimensionality after basic preprocessing on face and fingerprint images (highlighted). See Section~\ref{sec:preproc} for details.}
\label{tab:databases:experiments}
\begin{tabular}{clc}
\hline
\multirow{2}{*}{Modality}
& \multirow{2}{*}{Benchmark}
                                             & Dimensions \\
&                                            & $columns \times rows$ \\
\hline
\hline
\multirow{3}{*}{Iris}
&Warsaw~\cite{Czajka:MMAR:2013}              & $640 \times  480$ \\
&Biosec~\cite{Ruiz-Albacete:BIOID:2008}      & $640 \times  480$ \\
&MobBIOfake~\cite{Sequeira:VISAPP:2014:base} & $250 \times  200$ \\
\hline
\multirow{2}{*}{Face}
& Replay-Attack~\cite{Chakka:IJCB:2011}      & $\mathbf{200 \times  200}$ \\
& 3DMAD~\cite{Chingovska:ICB:2013}           & $\mathbf{200 \times  200}$ \\
\hline
\multirow{4}{*}{Fingerprint} 
&Biometrika~\cite{Ghiani:ICB:2013}           & $\mathbf{312 \times  372}$ \\
&CrossMatch~\cite{Ghiani:ICB:2013}           & $\mathbf{480 \times  675}$ \\
&Italdata~\cite{Ghiani:ICB:2013}             & $\mathbf{384 \times  432}$ \\
&Swipe~\cite{Ghiani:ICB:2013}                & $\mathbf{187 \times  962}$ \\
\hline

\end{tabular}
\end{center}
\end{table}

\subsubsection{Face Images}

Given that the face benchmarks considered in this work are video-based, we first evenly subsample 10 frames from each input video. Then, we detect the face position using Viola \& Jones~\cite{Viola:IJCV:2001} and crop a region of $200 \times 200$ pixels centered at the detected window.

\subsubsection{Fingerprint Images}
Given the diverse nature of images captured from different sensors, here the preprocessing is defined according to the sensor type.
\begin{enumerate}[(a)]
\item \emph{Biometrika}: we cropped the central region of size in columns and rows corresponding to 70\% of the original image dimensions. 
\item \emph{Italdata} and \emph{CrossMatch}: we cropped the central region of size in columns and rows respectively corresponding to 60\% and 90\% of the original image columns and rows.
\item \emph{Swipe}: As the images acquired by this sensor contain a variable number of blank rows at the bottom, the average number of non-blank rows $M$ was first calculated from the training images.
Then, in order to obtain images of a common size with non-blank rows, we removed their blank rows at the bottom and rescaled them to $M$ rows. Finally, we cropped the central region corresponding to 90\% of original image columns and $M$ rows.
\end{enumerate}

The rationale for these operations is based on the observation that fingerprint images in LivDet2013 tend to have a large portion of background content and therefore we try to discard such information that could otherwise mislead the representation learning process.
The percentage of cropped columns and rows differs among sensors because they capture images of different sizes with different amounts of background.

For architecture optimization (AO), the decision to use image color information was made according to 10-fold validation (see Section~\ref{sec:ao}), while for filter optimization (FO), color information was considered whenever available for a better approximation with the standard cf10-11 architecture. Finally, images were resized to $32\times32$ or $128\times128$ to be taken as input for the cf10-11 and spoofnet architectures, respectively. 

\subsection{Evaluation Protocol}
\label{sec:evalprot}

For each benchmark, we learn deep representations from their training images according to the methodology described in Section~\ref{sec:ao} for architecture optimization (AO) and in Section~\ref{sec:fo} for filter optimization (FO).
We follow the standard evaluation protocol of all benchmarks and evaluate the methods in terms of detection accuracy (ACC) and half total error rate (HTER), as these are the metrics used to assess progress in the set of benchmarks considered herein. Precisely, for a given benchmark and convolutional network already trained, results are obtained by:

\begin{enumerate}
\item Retrieving prediction scores from the testing samples;
\item Calculating a threshold $\tau$ above which samples are predicted as attacks;
\item Computing ACC and/or HTER using $\tau$ and test predictions.
\end{enumerate}

The way that $\tau$ is calculated differs depending on whether the benchmark has a development set or not (Table~\ref{tab:databases}). Both face benchmarks have such a set and, in this case, we simply obtain $\tau$ from the predictions of the samples in this set. 
Iris and fingerprint benchmarks have no such a set, therefore $\tau$ is calculated depending on whether the convolutional network was learned with AO or FO.

In case of AO, we calculate $\tau$ by joining the predictions obtained from 10-fold validation (see Section~\ref{sec:ao}) in a single set of positive and negative scores, and $\tau$ is computed as the point that lead to an equal error rate (EER) on the score distribution under consideration. 
In case of FO, scores are probabilities and we assume $\tau=0.5$. ACC and HTER are then trivially computed with $\tau$ on the testing set.

It is worth noting that the Warsaw iris benchmark provides a supplementary testing set that here we merge with the original testing set in order to replicate the protocol of~\cite{LivDet:Iris:2013}.
Moreover, given face benchmarks are video-based and that in our methodology we treat them as images (Section~\ref{sec:preproc}),
we perform a score-level fusion of the samples from the same video according to the max rule~\cite{Ross:HM:2006}. This fusion is done before calculating $\tau$.


\subsection{Implementation}
\label{sec:implementationdetais}

Our implementation for architecture optimization (AO) is based on Hyperopt-convnet~\cite{Bergstra:2013b} which in turn is based on Theano~\cite{Bergstra:SCIPY:2010}.
LibSVM~\cite{Chang:2011} is used for learning the linear classifiers via Scikit-learn.\footnote{http://scikit-learn.org}
The code for feature extraction runs on GPUs due to Theano and the remaining part is multithreaded and runs on CPUs.
We extended Hyperopt-convnet in order to consider the operations and hyperparameters as described in Appendix~\ref{sec:convnet_ops} and Section~\ref{sec:ao} and we will make the source code freely available in~\cite{Chiachia:2014b}.
Running times are reported with this software stack and are computed in an Intel i7 @3.5GHz with a Tesla K40 that, on average, takes less than one day to optimize an architecture --- i.e., to probe 2,000 candidate architectures --- for a given benchmark.

As for filter optimization (FO), Cuda-convnet~\cite{Krizhevsky:cuda-convnet:2012} is used. This library has an extremely efficient implementation to train convolutional networks via back-propagation on NVIDIA GPUs. 
Moreover, it provides us with the cf10-11 convolutional architecture taken in this work as reference for FO.

%%%%%%%%%%%%%%%%%%%%%%%%%%%%%%%%%%%%%%%%%%%%%%%%%%%%%%%%%%%%%%%%%%%%%%%%%
\begin{table*}[htb]
\begin{center}
\caption{
Overall results considering relevant information of the best found architectures, detection accuracy (ACC) and HTER values according to the evaluation protocol, and state-of-the-art (SOTA) performance.
}
\label{tab:resultsOARF}
%\tiny  \scriptsize \footnotesize \small \normalsize     
\iffinal
%\small
\else
\scriptsize
\fi
%               MD--TIME---ML--FEAT---L AH AH
\begin{tabular}{llr@{}c@{}l@{ }c@{ }c@{ }r@{ }c@{}l@{ }ccc@{ }ccc@{ }c@{ }c@{}c}
\hline
\multirow{3}{*}{modality}
& \multirow{3}{*}{benchmark} 
                & \multicolumn{9}{c}{architecture optimization (AO)}                     && \multicolumn{2}{c}{our results} && \multicolumn{3}{c}{SOTA results} & \\
                \cline{3-11} 
                \cline{13-14} \cline{16-18} 
&               &  \multicolumn{3}{c}{time}
                           & size 
                                 & layers
                                     & \multicolumn{3}{c}{features} 
                                                                         & objective 
                                                                                  && ACC & HTER && ACC & HTER & \multirow{2}{*}{Ref.}\\
&               & \multicolumn{3}{c}{(secs.)}
                          & (pixels) 
                                 &$(\#)$
                                     & \multicolumn{3}{c}{$(\#)$} 
                                                                          &  $(\%)$    &&$(\%)$ &$(\%)$ &&$(\%)$  & $(\%)$ & \\
\hline\hline
\multirow{2}{*}{iris}
& Warsaw            &  52&+&35 & 640 & 2 & $10\times 15\times  64$ &&  (9600) & 98.21 && \textbf{99.84}
                                                                                            &  0.16 && 97.50  & ---    & \cite{Czajka:MMAR:2013} \\ % *0.0/5.0 MMAR 2013, LivDet'2013 - Iris competion - *5.25%/11.95%  
& Biosec        &  80&+&34 & 640 & 3 & $ 2\times  5\times 256$ &&  (2560) & 97.56 &&  98.93 &  1.17 && \textbf{100.00}
                                                                                                              & ---    & \cite{Galbally:ICB:2012} \\ % ICB 2012 / VISAPP  - IJCANN Porto (99.63%)   
& MobBIOfake    &  18&+&37 & 250 & 2 & $ 5\times  7\times 256$ &&  (8960) & 98.94 &&  98.63 &  1.38 && \textbf{99.75}
                                                                                                                       & ---    & \cite{Sequeira:IJCB:2014} \\ % $0.5/0.0 - MobILive 2014 Competition    
\hline                                      
\multirow{2}{*}{face}
& Replay-Attack &  69&+&15 & 256 & 2 & $ 3\times  3\times 256$ &&  (2304) & 94.65 &&  98.75 &  \textbf{0.75} && ---    &   5.11 & \cite{Komulainen:ICB:2013} \\ %Competition ICB'2013  9.12%   -   0.00%       
& 3DMAD         &  55&+&15 & 128 & 2 & $ 5\times  5\times  64$ &&  (1600) & 98.68 && 100.00 &  \textbf{0.00}
                                                                                                    && ---    &   0.95 & \cite{Erdogmus:BIOSIG:2013}\\ 
\hline
\multirow{2}{*}{fingerprint}
& Biometrika    &  66&+&25 & 256 & 2 & $ 2\times  2\times 256$ &&  (1024) & 90.11 &&  96.50 &  3.50 &&  \textbf{98.30}
                                                                                                                       & ---    & \cite{Ghiani:ICB:2013} \\ % LivDet'2013 - Dermalog     
& Crossmatch    & 112&+&12 & 675 & 3 & $ 2\times  3\times 256$ &&  (1536) & 91.70 && \textbf{92.09} 
                                                                                            &  8.44 &&  68.80 & ---    & \cite{Ghiani:ICB:2013} \\ % LivDet'2013 - UniNap1  
& Italdata      &  46&+&27 & 432 & 3 & $16\times 13\times 128$ && (26624) & 86.89 &&  97.45 &  2.55 &&  \textbf{99.40}
                                                                                                                       & ---    & \cite{Ghiani:ICB:2013} \\ % LivDet'2013 - Anonymous2   
& Swipe         &  97&+&51 & 962 & 2 & $53\times  3\times  32$ &&  (5088) & 90.32 &&  88.94 & 11.47 &&  \textbf{96.47}
                                                                                                                       & ---    & \cite{Ghiani:ICB:2013} \\ % LivDet'2013 - Dermalog     
\hline
\end{tabular}
\end{center}
\end{table*}
%%%%%%%%%%%%%%%%%%%%%%%%%%%%%%%%%%%%%%%%%%%%%%%%%%%%%%%%%%%%%%%%%%%%%%%%%%%

\section{Experiments and Results}
\label{sec:experiments}



In this section, we evaluate the effectiveness of the proposed methods for spoofing detection. We show experiments for the architecture optimization and filter learning approaches along with their combination  
for detecting iris, face, and fingerprint spoofing on the nine benchmarks described in Section~\ref{sec:databases}. We also present results for the \emph{spoofnet}, which incorporates some domain-knowledge on the problem. We compare all of the results with the state-of-the-art counterparts. Finally, we discuss the pros and cons of using such approaches and their combination along with efforts to understand the type of features learned and some effeciency questions when testing the proposed methods.



\subsection{Architecture Optimization (AO)}

Table~\ref{tab:resultsOARF} presents AO results in detail as well as previous state-of-the-art (SOTA) performance for the considered benchmarks. With this approach, we can outperform four SOTA methods in all three biometric modalities. Given that AO assumes little knowledge about the problem domain, this is remarkable. Moreover, performance is on par in other four benchmarks, with the only exception of Swipe. Still in Table~\ref{tab:resultsOARF}, we can see information about the best architecture such as time taken to evaluate it (feature extraction + 10-fold validation), input size, depth, and dimensionality of the output representation in terms of \emph{columns} $\times$ \emph{rows} $\times$ \emph{feature maps}.

Regarding the number of layers in the best architectures, we can observe that six out of nine networks use two layers, and three use three layers. We speculate that the number of layers obtained is a function of the problem complexity.
In fact, even though there are many other hyperparameters involved, the number of layers play an important role in this issue, since it directly influences the level of non-linearity and abstraction of the output with respect to the input.

With respect to the input size, we can see in comparison with Table~\ref{tab:databases:experiments}, that the best performing architectures often use the original image size. This was the case for all iris benchmarks and for three (out of four) fingerprint benchmarks. For face benchmarks, a larger input was preferred for Replay-Attack, while a smaller input was preferred for 3DMAD. We hypothesize that this is also related to the problem difficulty, given that Replay-Attack seems to be more difficult, and that larger inputs tend to lead to larger networks.

We still notice that the dimensionality of the obtained representations are, in general, smaller than 10K features, except for Italdata.
Moreover, for the face and iris benchmarks, it is possible to roughly observe a relationship between the optimization objective calculated in the training set and the detection accuracy measure on the testing set (Section~\ref{sec:evalprot}), which indicates the appropriateness of the objective for these tasks. However, for the fingerprint benchmarks, this relationship does not exist, and we accredit this to either a deficiency of the optimization objective in modelling these problems or to the existence of artifacts in the training set misguiding the optimization.

% Table~\ref{tab:resultsOARF} shows the overall results and is divided in three parts.
% We can see that the performance of proposed methods are competitive. In fact, by training simple linear classifiers on the deep representations found with the architecture optimization approach, the proposed methods outperform four SOTA methods and are on par with other four methods. 
% In particular, we obtained state-of-the-art performance on the Warsaw (iris), Replay-Attack (face), 3DMAD (face), and CrossMatch (fingerprint) benchmarks, at least one in each biometric modality. When considering architecture optimization, the less compelling result was for the Swipe benchmark.

% Analysing the first part of Table~\ref{tab:resultsOARF}, we can observe important characteristics of the best CNNs found in the architecture hyperparameter optimization step such as the time (in seconds) taken to evaluate it (feature extraction + 10-run validation), the input size that the CNN assumes ($\max_{axis}$), its depth ($d_{choice}$), and the dimensionality of its output representations in terms of \emph{columns} $\times$ \emph{rows} $\times$ \emph{bands}.

% Taking into account that understanding how a set of deep learned features capture properties and nuances of a problem is still an open question in the vision community, in the following we discuss the listed properties of the optimized CNNs that may bring clues and insights on how these features are meaningful for the spoofing problem.

% Regarding the number of layers required by the CNNs producing the best representations, we can observe that six out of nine CNNs use two layers, and three CNNs use three layers.
% We speculate that the number of layers obtained is a function of the problem complexity.
% Even though there are many other hyperparameters involved, the number of layers play an important role in this issue, since it directly influences the level of non-linearity and abstraction of the output with respect to the input.

% \TODO{This is still true? Before you mentioned for the spoofnet that wanted larger images as a domain-knowledge to incorporate more information. This is not in line with that}
% Regarding input image size ($\max_{axis}$), in comparison with Table~\ref{tab:databases:experiments}, we can see that the best performing CNNs often use the original image size. This was the case for all iris benchmarks and for three (out of four) fingerprint benchmarks. For the face benchmarks, a larger input was preferred for Replay-Attack, while a smaller input was preferred for 3DMAD. We hypothesize that this is also related to the problem difficulty, given that Replay-Attack seems to be more difficult, and that larger inputs tend to lead to larger networks.

% Still with respect to the outcomes of the optimization, we notice that the dimensionality of the obtained representations are, in general, smaller than 10K features, except for Italdata.
% Moreover, for the face and iris benchmarks, it is possible to roughly observe a relationship between the optimization objective calculated in the training set and the detection accuracy measure on the testing set (Section~\ref{sec:evalprot}), which indicates the appropriateness of the objective for these tasks. However, for the fingerprint benchmarks, this relationship does not exists, and we accredit this to either a deficiency of the optimization objective in modeling these problems or the existence of artifacts in the training set misguiding the optimization.


\subsection{Filter Optimization (FO)}

\begin{table}
\begin{center}
\caption{
Results for filter optimization (FO) in \emph{cf10-11} and \emph{spoofnet} (Fig.~\ref{fig:cudaconvnet}). Even though both networks present similar behavior, \emph{spoofnet} is able to push performance even further in problems which \emph{cf10-11} was already good for.
Architecture optimization (AO) results (with random filters) are shown in the first column to facilitate comparisons.
}
\label{tab:FO}
\begin{tabular}{c@{ }c@{ }l@{ }r@{ }r@{ }r@{ }r@{ }r@{ }r@{ }r@{ }r@{ }c}
\hline
% \multirow{2}{*}{benchmark} 
         & \hspace{1em} && \multicolumn{6}{c}{filter}\\ \cline{4-9}
modality & \hspace{1em} &&& random && \multicolumn{3}{c}{optimized} \\ \cline{5-5} \cline{7-9}
(metric) && benchmark \hspace{1em} && \multicolumn{1}{c}{AO} && \textit{cf10-11} && \textit{spoofnet} && SOTA & \\
\hline
iris && Warsaw        && \textbf{99.84}  && 67.20 &&          66.42  &&          97.50 \\
(ACC)&& Biosec        &&         98.93   && 59.08 &&          47.67  && \textbf{100.00} \\
     && MobBIOfake    &&         98.63   && 99.13 && \textbf{100.00} &&          99.75 \\
\hline
face && Replay-Attack &&\textbf{0.75} && 55.13 && 55.38 && 5.11\\
(HTER) && 3DMAD         &&\textbf{0.00} && 40.00 && 24.00 && 0.95\\ 
\hline
fingerprint &&Biometrika    && 96.50 && 98.50&& \textbf{99.85} && 98.30\\
(ACC)     &&Crossmatch    && 92.09 && 97.33&& \textbf{98.23} && 68.80\\
     &&Italdata      && 97.45 && 97.35&& \textbf{99.95} && 99.40\\
     &&Swipe         && 88.94 && 98.70&& \textbf{99.08} && 96.47\\
\hline
\end{tabular}
\end{center}
\end{table}

% \begin{table}
% \red{
% \begin{center}
% \caption{\TODO{write me}}
% \label{tab:FO}
% \begin{tabular}{l@{ }c@{ }c@{ }c@{ }c@{ }c@{ }c@{ }c@{ }c@{ }c@{ }c@{ }c@{ }c}
% \hline
% \multirow{2}{*}{benchmark} 
%                 && \multicolumn{2}{c}{OA} && \multicolumn{2}{c}{\textit{cf10-11}} && \multicolumn{2}{c}{\textit{spoofnet}} && SOTA & \\
%                 \cline{3-4} \cline{6-7} \cline{9-10} 
%                 \cline{12-12}
%                &&$(\%)$&$(\%)$&&$(\%)$&$(\%)$&&$(\%)$&$(\%)$&&$(\%)$\\
% \hline\hline
% Warsaw        && \textbf{99.84}  & 59.55 && 87.06 & 67.20 && 96.44 &          66.42  &&          97.50 \\
% Biosec        &&         98.93   & 57.50 && 97.33 & 59.08 && 97.42 &          47.67  && \textbf{100.00} \\
% MobBIOfake    &&         98.63   & 99.38 && 77.00 & 99.13 && 72.00 & \textbf{100.00} &&          99.75 \\
% \hline
% Replay-Attack &&\textit{\textbf{0.75}} & \textit{55.88} &&\textit{5.62} &\textit{55.13} &&\textit{3.50} &\textit{55.38} &&\textit{5.11}\\
% 3DMAD         &&\textit{\textbf{0.00}} & \textit{40.00} &&\textit{8.00} &\textit{ 40.00}&&\textit{4.00} &\textit{ 24.00}&&\textit{0.95}\\ 
% \hline
% Biometrika    && 96.50& 99.30&& 77.45& 98.50&& 94.70&\textbf{99.85} && 98.30\\
% Crossmatch    && 92.09& 98.04&& 83.11& 97.33&& 87.82&\textbf{98.23} && 68.80\\
% Italdata      && 97.45& 99.45&& 76.45& 97.35&& 91.05&\textbf{99.95} && 99.40\\
% Swipe         && 88.94& 99.08&& 87.60& 98.70&& 96.75&\textbf{99.08} && 96.47\\
% \hline
% \end{tabular}
% \end{center}
% Abbreviations: \emph{OA} - Optimized Architecture; \emph{Arc.} - Architecture; \emph{RF} - random filters; \emph{LF} - learned filter through backpropagation
% }
% \end{table}


Table~\ref{tab:FO} shows the results for FO, where we repeat architecture optimization (AO) results (with random filters) in the first column to facilitate comparisons. Overall, we can see that both networks, \emph{cf10-11} and \emph{spoofnet} have similar behavior across the biometric modalities.

Surprisingly, \emph{cf10-11} obtains excellent performance in all four fingerprint benchmarks as well as in the MobBIOFake, exceeding SOTA in three cases, in spite of the fact that it was used without any modification. However, in both face problems and in two iris problems, \emph{cf10-11} performed poorly. Such difference in performance was not possible to anticipate by observing training errors, which steadily decreased in all cases until training was stopped. Therefore, we believe that in these cases FO was misguided by the lack of training data or structure in the training samples irrelevant to the problem.
% Even though they perform both poorly in Warsaw, Biosec, Replay-Attack and 3DMAD, they

%% THIS PART BELOW GOES TO THE TEXT AFTER THE SSS CLAIM
\rva{To reinforce this claim, we performed experiments with filter optimization (FO) in \emph{spoofnet} by varying the training set size with 20\%, 40\%, and 50\% of fingerprint benchmarks. As expected, in all cases, the less training examples, the worse is the generalization of the \emph{spoofnet} (lower classification accuracies). 
Considering the training phase, for instance, when using 50\% of training set or less, the accuracy achieved by the learned representation is far worse than the one achieved when using 100\% of training data. 
This fact reinforces the conclusion presented herein regarding the small sample size problem. 
Maybe a fine-tuning of some parameters, such as the number of training epochs and the learning rates, can diminish the impact of the small sample size problem stated here, however, this is an open research topic by itself.}

For \emph{spoofnet}, the outcome is similar. As we expected, the proposed architecture was able to push performance even further in problems which \emph{cf10-11} was already good for, outperforming SOTA in five out of nine benchmarks. This is possibly because we made the \emph{spoofnet} architecture simpler, with less parameters, and taking input images with a size better suited to the problem.

As compared to the results in AO, we can observe a good balance between the approaches. In AO, the resulting convolutional networks are remarkable in the face benchmarks. In FO, networks are remarkable in fingerprint problems. While in AO all optimized architectures have good performance in iris problems, FO excelled in one of these problems, MobBIOFake, with a classification accuracy of 100\%. In general, AO seems to result in convolutional networks that are more stable across the benchmarks, while FO shines in problems in which learning effectively occurs. Considering both AO and FO, we can see in Table~\ref{tab:FO} that we outperformed SOTA methods in eight out of nine benchmarks. The only benchmark were SOTA performance was not achieved is Biosec, but even in this case the result obtained with AO is competitive. 

%%\TODO{Allan's figure discussion starts here.}
%\input{fig_allan}
Understanding how a set of deep learned features capture properties and nuances of a problem is still an open question in the vision community. However, in an attempt to understand the behavior of the operations applied onto images after they are forwarded through the first convolutional layer, we generate Fig.~\ref{fig:filter_learned_conv1} that illustrates the filters learned via backpropagation algorithm and Figs.~\ref{fig:mean_pos_class}~and~\ref{fig:mean_neg_class} showing the mean of real and fake images that compose the test set, respectively. To obtain output values from the first convolutional layer and get a sense of them, we also instrumented the \textit{spoofnet} convolutional network to forward the real and fake images from the test set through network. Figs~\ref{fig:act_map_class_1}~and~\ref{fig:act_map_class_0} show such images for the real and fake classes, respectively.

We can see in Fig.~\ref{fig:filter_learned_conv1} that the filters learned patterns resemble textural patterns instead of edge patterns as usually occurs in several computer vision problems~\cite{Krizhevsky:2012,Ouyang:2014}. This is particularly interesting and in line with several anti-spoofing methods in the the literature which also report good results when exploring texture information~\cite{Ghiani:ICB:2013, Maatta:IJCB:2011}.

In addition, Fig.~\ref{fig:mean_pos_class}~and~\ref{fig:mean_neg_class} show there are differences between real and fake images from test, although apparently small in such a way that a direct analysis of the images would not be enough for decision making. However, when we analyze the mean activation maps for each class, we can see more interesting patterns. In Figs.~\ref{fig:act_map_class_1}~and~\ref{fig:act_map_class_0}, we have sixteen pictures with $128 \times 128$ pixel resolution. These images correspond to the sixteen filters that composing the first layer of the \textit{spoofnet}. Each position $(x,y)$ in these $128 \times 128$ images corresponds to a $5 \times 5$ area (receptive field units) in the input images. Null values in a given unit means that the receptive field of the unit was not able to respond to the input stimuli. In contrast, non-null values mean that the receptive field of the unit had a responsiveness to the input stimuli.

We can see that six filters have a high responsiveness to the background information of the input images (filters predominantly white) whilst ten filters did not respond to background information (filters predominantly black). From left to right, top to bottom, we can see  also that the images corresponding to the filters 2, 7, 13, 14 and 15 have high responsiveness to information surrounding the central region of the sensor (usually where fingerprints are present) and rich in texture datails. Although these regions of high and low responsiveness are similar for both classes we can notice some differences. A significant difference in this first convolutional layer to images for the different classes is that the response of the filters regarding to fake images (Fig~\ref{fig:act_map_class_0}) generates a blurring pattern, unlike the responses of the filters regarding to real images (Fig~\ref{fig:act_map_class_1}) which generate a sharper pattern. We believe that the same way as the first layer of a convolutional network has the ability to respond to simple and relevant patterns (edge information) to a problem of recognition objects in general, in computer vision, the first layer in the \textit{spoofnet} also was able to react to a simple pattern recurrent in spoof problems, the blurring effect, an artifact previously explored in the literature~\cite{Galbally:TIP:2014}. Finally, we are exploring visualisation only of the first layer; subsequent layers of the network can find new patterns in these regions activated by the first layer further emphasizing class differences. 

\begin{figure}
\centering
\subfloat[Filter weights of the first convolutional layer that were learned using the backpropagation algorithm.]{\includegraphics[width=0.46\textwidth]{crossmatch_conv1_without_background.pdf}\label{fig:filter_learned_conv1}}\\
\subfloat[Mean for real images (test set).]{\includegraphics[width=0.20\textwidth]{crossmatch_test_pos_class_with_border.pdf}\label{fig:mean_pos_class}}\hspace{1mm}
\subfloat[Mean for fake images (test set).]{\includegraphics[width=0.20\textwidth]{crossmatch_test_neg_class_with_border.pdf}\label{fig:mean_neg_class}}\\
\subfloat[Activation maps for real images.]{\includegraphics[width=0.23\textwidth]{crossmatch_activation_map_class_1.pdf}\label{fig:act_map_class_1}}\hspace{1mm}
\subfloat[Activation maps for fake images.]{\includegraphics[width=0.23\textwidth]{crossmatch_activation_map_class_0.pdf}\label{fig:act_map_class_0}}\\

\caption{Activation maps of the filters that compose the first convolutional layer when forwarding real and fake images through the network.}
\end{figure}

% The last round of experiments evaluates the \emph{spoofnet} architecture, which incorporates our domain-knowledge on the problem taken from observations of the architecture optimization experience.
% That is, given that we are working on a relatively simple and binary problem different from the traditional multiclass problems present in vision, a small and less complex network could be able to accommodate the variability of the spoofing problem more effectively.
% Moreover, the use of a large input image could also be beneficial to obtain better features of the recapture process inherent to spoofing such as the ones related to the added noise signatures during the process.
% We compare such network considering random and learned filter weights and show results on Table~\ref{tab:compAll}.

% In this scenario, the \textit{spoofnet} has systematically outperformed \textit{cf10-11}'s network and the optimized architecture in the benchmarks for which filter learning approach performed well in previous experiments.

% Moreover, \textit{spoofnet} outperforms the SOTA results in five out nine  benchmarks, and the architecture optimization approach outperforms the SOTA results in another three benchmarks.

% That is, the combination of the proposed methods achieves new SOTA results in eight out nine cases, three SOTA results when optimizing the architecture and using random filters and five SOTA results when incorporating domain-knowledge in the design of a network using learned filters.
% The only case in which the use of convolutional networks as proposed here does not achieve SOTA results is for the Biosec benchmark. 
% In this case, the proposed method using architecture optimization provides a result of about 1\% lower than the SOTA, which is still a remarkable accomplishment (98.93\% v. 100.0\% classification accuracy).

% The final remark is that the filter learning approach has always failed to four benchmarks: Warsaw, Biosec, Replay-Attack, and 3DMAD.
% Our hypotheses for such fact are threefold:  
% (1) the existence of few samples available, which hinders the learning process;  
% (2) the lack of variability (subjects, sessions, etc.) in the training set limiting the generalization of the learned representation; and 
% (3) particular nuances existing in the testing set not present in the training set (e.g., due to different session acquisition conditions) that could not be captured in the learning process but could be incorporated by the architecture optimization approach.



% Tables~\ref{tab:resultsOARF},~\ref{tab:compOAAA}, and~\ref{tab:compAll} present the effectiveness results for spoofing detection considering the proposed methods with respect to the state of the art. 
% In Tables~\ref{tab:compOAAA} and~\ref{tab:compAll} we present the accuracy values, except when they are in \emph{italic}. When in italic, the value refers to HTER values, the standard evaluation measure for Replay-Attack and 3DMAD benchmarks.

% We now turn our attention to our second set of contributions, more specifically, when considering an approach for learning filter weights on a given network. Given a well-known architecture in a given scenario (e.g., vision problems), such as \textit{cf10-11}, we learn its filter weights using the methodology described in Section~\ref{sec:fwl_training}.
% Then, we compare their effectiveness with the ones of the architecture optimization.

% Two important aspects can be observed here. First, the filter learning approach (\textit{cf10-11}+LF) outperforms the existing SOTA results on three out of nine benchmarks, that is, Biometrika, Crossmatch and Swipe -- all fingerprint benchmarks. It also outperforms the architecture optimization approach (OA+RF) on another benchmark, MobBIOfake (iris).
% The effectiveness achieved by the filter learning approach (\textit{cf10-11}+LF) in four benchmarks (Warsaw, Biosec, Replay-Attack and 3DMAD), on the other hand, might be surprising as the network here was learned for a vision problem and only its filters were learned for spoofing problems.


\subsection{Interplay between AO and FO}

\begin{table}
\begin{center}
\caption{Results for architecture and filter optimization (AO+FO) along with \emph{cf10-11} and \emph{spoofnet} networks considering random weights.
AO+FO show compelling results for fingerprints and one iris benchmark (MobBIOFake). We can also see that \emph{spoofnet} can benefit from random filters in situations it was not good for when using filter learning (e.g., Replay-Attack).}
\label{tab:interplay}
\begin{tabular}{c@{ }c@{ }l@{ }r@{ }c@{ }r@{ }r@{ }r@{ }r@{ }r@{ }r@{ }c}
\hline
% \multirow{2}{*}{benchmark} 
         & \hspace{1em} && \multicolumn{6}{c}{filter}\\ \cline{4-9}
modality & \hspace{1em} &&& optimized && \multicolumn{3}{c}{random} \\ \cline{5-5} \cline{7-9}
(metric) && benchmark \hspace{1em} && AO && \textit{cf10-11} && \textit{spoofnet} && SOTA & \\
\hline
iris && Warsaw        &&                59.55  && 87.06 &&          96.44  && \textbf{ 97.50} \\
(ACC)&& Biosec        &&                57.50  && 97.33 &&          97.42  && \textbf{100.00} \\
     && MobBIOfake    &&                99.38  && 77.00 &&          72.00  &&  \textbf{ 99.75} \\
\hline
face && Replay-Attack &&                55.88  &&  5.62 &&  \textbf{ 3.50} &&           5.11 \\
(HTER) && 3DMAD       &&                40.00  &&  8.00 &&           4.00  &&  \textbf{ 0.95} \\ 
\hline
fingerprint &&Biometrika     && \textbf{99.30}  && 77.45 &&          94.70  &&           98.30\\
(ACC)     &&Crossmatch       && \textbf{98.04}  && 83.11 &&          87.82  &&           68.80\\
     &&Italdata              && \textbf{99.45}  && 76.45 &&          91.05  &&           99.40\\
     &&Swipe                 && \textbf{99.08}  && 87.60 &&          96.75  &&           96.47\\
\hline
\end{tabular}
\end{center}
\end{table}

In the previous experiments, architecture optimization (AO) was evaluated using random filters and filter optimization (FO) was carried out in the predefined architectures \emph{cf10-11} and \emph{spoofnet}. A natural question that emerges in this context is how these methods would perform if we (i) combine AO and FO and if we (ii) consider random filters in \emph{cf10-11} and \emph{spoofnet}.

Results from these combinations are available in Table~\ref{tab:interplay} and show a clear pattern. When combined with AO, FO again exceeds previous SOTA in all fingerprint benchmarks and performs remarkably good in MobBIOFake. However, the same difficulty found by FO in previous experiments for both face and two iris benchmarks is also observed here.
Even though \emph{spoofnet} performs slightly better than AO in the cases where SOTA is exceeded (Table~\ref{tab:FO}), it is important to remark that our AO approach may result in architectures with a much larger number of filter weights to be optimized, and this may have benefited~\emph{spoofnet}.

It is also interesting to observe in Table~\ref{tab:interplay} the results obtained with the use of random filters in \emph{cf10-11} and \emph{spoofnet}. The overall balance in performance of both networks across the benchmarks is improved, similar to what we have observed with the use of random filters in Table~\ref{tab:resultsOARF}. An striking observation is that~\emph{spoofnet} with random filters exceed previous SOTA in Replay-Attack, and this supports the idea that the poor performance of~\emph{spoofnet} in Replay-Attack observed in the FO experiments (Table~\ref{tab:FO}) was not a matter of architecture.


% Second, the \textit{cf10-11} network when using random filters (\textit{cf10-11}+RF) shows a relatively high effectiveness when compared to the learning filter approach (\textit{cf10-11}+LF) on four datasets where the learning approach clearly fails (Warsaw, Biosec, Replay-Attack, 3DMAD).

% Finally, another interesting observation: the optimized architectures with learned filters (OA+LF) achieved better classification results than the ones of \textit{cf10-11} whenever the \textit{cf10-11}+LF worked well. In such cases, the combination of architecture and filter optimization approaches outperformed the SOTA results in seven out of nine benchmarks.

% The \textit{spoofnet} with random filters has achieved promising results in almost all benchmarks (except for MobBIOfake), and has outperformed the previous SOTA results for the Crossmatch and Swipe benchmarks. That is, if someone does not have enough data or time to train an optimized architecture, this solution be a good tradeoff. 


% \begin{table}[htb]
% \red{
% \begin{center}
% \caption{The effects of combining existing (cf10-11) and optimized architectures (OA) with filter learning (LF) and random filters (RF).}
% \label{tab:compOAAA}
% %\tiny  \scriptsize \footnotesize \small \normalsize     
% \iffinal
% %\small
% \else
% \scriptsize
% \fi
% %               MD--TIME---ML--FEAT---L AH AH
% \begin{tabular}{l@{ }c@{ }c@{ }c@{ }c@{ }c@{ }c@{ }c@{ }c@{ }c}
% \hline
% \multirow{3}{*}{Database} 
%                 && \multicolumn{2}{c}{OA} 
%                                                       && \multicolumn{2}{c}{\textit{cf10-11}} 
%                                                                                            && SOTA & \\
                
%                 \cline{3-4} \cline{6-7} 
%                && RF & LF && RF & LF && Results & \\
%                                                                \cline{9-9}      
%                &&$(\%)$&$(\%)$&&$(\%)$&$(\%)$&&$(\%)$\\
% \hline\hline
% Warsaw        && \textbf{99.84} 
%                       & 59.55&& 87.06& 67.20&& 97.50\\
% Biosec        && 98.93& 57.50&& 97.33& 59.08&&\textbf{100.00} \\
% MobBIOfake    && 98.63&\multicolumn{1}{>{\columncolor[gray]{.8}[9pt]}c}{99.38}
%                              && 77.00&\multicolumn{1}{>{\columncolor[gray]{.8}[9pt]}c}{99.13}
%                                             && \textbf{99.75} \\
% \hline
% Replay-Attack &&\textit{\textbf{0.75}}
%                         & \textit{55.88}
%                                &&\textit{5.62}
%                                        &\textit{55.13}
%                                             &&\textit{5.11}\\
% 3DMAD         &&\textit{\textbf{0.00}}
%                         &\textit{40.00}
%                              &&\textit{  8.00}
%                                      &\textit{ 40.00}
%                                             &&\textit{  0.95}\\ 
% \hline
% Biometrika    && 96.50& \textbf{99.30}
%                              && 77.45&\multicolumn{1}{>{\columncolor[gray]{.8}[9pt]}c}{98.50}
%                                             && 98.30\\ % LivDet'2013 - Dermalog   
% Crossmatch    && 92.09& \textbf{98.04}
%                              && 83.11&\multicolumn{1}{>{\columncolor[gray]{.8}[9pt]}c}{97.33}
%                                             && 68.80\\ % LivDet'2013 - UniNap1    
% Italdata      && 97.45& \textit{99.45}
%                              && 76.45& 97.35&& 99.40\\ % LivDet'2013 - Anonymous2     
% Swipe         && 88.94& \textbf{99.08}
%                              && 87.60&\multicolumn{1}{>{\columncolor[gray]{.8}[9pt]}c}{98.70}
%                                             && 96.47\\ %  
% \hline
% \end{tabular}
% \end{center}
% Abbreviations: \emph{OA} - optimized architecture; \emph{RF} - random filters; \emph{LF} - learned filter through backpropagation\\
% }
% \end{table}

% \begin{table}[htb]
% \red{
% \begin{center}
% \caption{The effects of filter learning (LF) and random filters (RF) on existing (cf10-11) and optimized architectures (OA) as well as the advantages of incorporating domain-knowledge in the architecture design (spoofnet).}
% \label{tab:compAll}
% %\tiny  \scriptsize \footnotesize \small \normalsize     
% \iffinal
% %\small
% \else
% \scriptsize
% \fi
% %               MD--TIME---ML--FEAT---L AH AH
% \begin{tabular}{l@{ }c@{ }c@{ }c@{ }c@{ }c@{ }c@{ }c@{ }c@{ }c@{ }c@{ }c@{ }c}
% \hline
% \multirow{3}{*}{Database} 
%                 && \multicolumn{2}{c}{OA} 
%                                                       && \multicolumn{2}{c}{\textit{cf10-11}}
%                                                                                            && \multicolumn{2}{c}{\textit{spoofnet}} &&
%                                                                                            SOTA & \\
                
%                 \cline{3-4} \cline{6-7} \cline{9-10} 
%                && RF & LF && RF & L.F. && RF & LF && Results & \\
%                                                                \cline{12-12}      
%                &&$(\%)$&$(\%)$&&$(\%)$&$(\%)$&&$(\%)$&$(\%)$&&$(\%)$\\
% \hline\hline
% Warsaw        && \textbf{99.84} 
%                         & 59.55&& 87.06& 67.20&& 96.44& 66.42&& 97.50\\
% Biosec        &&\multicolumn{1}{>{\columncolor[gray]{.8}[9pt]}c}{ 98.93}
%                       & 57.50&& 97.33& 59.08&& 97.42& 47.67&&\textbf{100.00} \\
% MobBIOfake    && 98.63& 99.38&& 77.00& 99.13&& 72.00&\textbf{100.00} 
%                                                              && 99.75 \\
% \hline
% %\rowcolor{LightCyan}
% Replay-Attack &&\textit{\textbf{0.75}}
%                         & \textit{55.88}
%                                &&\textit{5.62}
%                                        &\textit{55.13}
%                                               &&\textit{3.50}
%                                                       &\textit{55.38}
%                                                              &&\textit{5.11}\\
% %\rowcolor{LightCyan}                               
% 3DMAD         &&\textit{\textbf{0.00}}
%                         &\textit{40.00}
%                              &&\textit{  8.00}
%                                      &\textit{ 40.00}
%                                             &&\textit{  4.00}
%                                                     &\textit{ 24.00}
%                                                              &&\textit{  0.95}\\ 
% \hline
% Biometrika    && 96.50& 99.30&& 77.45& 98.50&& 94.70&\textbf{99.85}
%                                                              && 98.30\\ % LivDet'2013 - Dermalog  
% Crossmatch    && 92.09& 98.04&& 83.11& 97.33&& 87.82&\textbf{98.23} 
%                                                              && 68.80\\ % LivDet'2013 - UniNap1   
% Italdata      && 97.45& 99.45&& 76.45& 97.35&& 91.05&\textbf{99.95} 
%                                                              && 99.40\\ % LivDet'2013 - Anonymous2    
% Swipe         && 88.94& 99.08&& 87.60& 98.70&& 96.75&\textbf{99.08}
%                                                              && 96.47\\ %     
% \hline
% \end{tabular}
% \end{center}
% Abbreviations: \emph{OA} - Optimized Architecture; \emph{Arc.} - Architecture; \emph{RF} - random filters; \emph{LF} - learned filter through backpropagation
% }
% \end{table}


\subsection{Runtime}

We estimate time requirements for anti-spoofing systems built with convolutional networks based on measurements obtained in architecture optimization (AO).
We can see in~Table~\ref{tab:resultsOARF} that the most computationally intensive deep representation is the one found for the Swipe benchmark, and demands 148 (97+51) seconds to process 2,200 images. Such a running time is only possible due to the GPU+CPU implementation used (Section~\ref{sec:implementationdetais}), which is critical for this type of learning task. In a hypothetical operational scenario, we could ignore the time required for classifier training (51 seconds, in this case). Therefore, we can estimate that, on average, a single image captured by a Swipe sensor would require approximately 45 milliseconds --- plus a little overhead --- to be fully processed in this hypothetical system. Moreover, the existence of much larger convolutional networks running in realtime in budgeted mobile devices~\cite{Wardern:2014} also supports the idea that the approach is readily applicable in a number of possible scenarios.
% The first thing to notice in Table~\ref{tab:results} is the time required to evaluate the best CNNs on the respective benchmark training sets.


\begin{figure*}[!t]
\begin{center}
\includegraphics[width=0.85\linewidth]{datasets.pdf}
\caption{Examples of hit and missed testing samples lying closest to the real-fake decision boundary of each benchmark. A magnified visual inspection on these images may suggest some properties of the problem to which the learned representations are sensitive.}
\label{fig:datasets}\label{page:fig:datasets}
\end{center}
\end{figure*}


\subsection{Visual Assessment}

In Fig.~\ref{fig:datasets}, we show examples of hit and missed testing samples lying closest to the real-fake decision boundary of the best performing system in each benchmark. 
A magnified visual inspection on these images may give us some hint about properties of the problem to which the learned representations are sensitive.

While it is difficulty to infer anything concrete, it is interesting to see that the real missed sample in Biosec is quite bright, and that skin texture is almost absent in this case. 
% In contrast, we have achieved perfection classification for MobBIOfake database.
%Still, we may argue that a possible reason for the fake miss in MobBIOfake is the use of glasses, 
Still, we may argue that a noticeable difference exists in Warsaw between the resolution used to print the images that led to the fake hit and the fake miss.

Regarding the face benchmarks, the only noticeable observation from Replay-Attack is that the same person is missed both when providing to the system a real and a fake biometric reading. This may indicate that some individuals are more likely to successfully attack a face recognition systems than others. In 3DMAD, it is easy to see the difference between the real and fake hits. Notice that there was no misses in this benchmark.

A similar visual inspection is much harder in the fingerprint benchmarks, even though the learned deep representations could effectively characterize these problems. The only observation possible to be made here is related to the fake hit on CrossMatch, which is clearly abnormal.
The images captured with the Swipe sensor are naturally narrow and distorted due to the process of acquisition, and this distortion prevents any such observation.
%\newpage
\section{Conclusions and Future Work}
\label{sec:conclusions}

In this work, we investigated two deep representation research approaches for detecting spoofing in different biometric modalities. On one hand, we approached the problem by learning representations directly from the data through architecture optimization with a final decision-making step atop the representations. On the other, we sought to learn filter weights for a given architecture using the well-known back-propagation algorithm. As the two approaches might seem naturally connected, we also examined their interplay when taken together. In addition, we incorporated our experience with architecture optimization as well as with training filter weight for a given architecture into a more interesting and adapted network, \emph{spoofnet}.  

Experiments showed that these approaches achieved outstanding classification results for all problems and modalities outperforming the state-of-the-art results in eight out of nine benchmarks.
Interestingly, the only case for which our approaches did not achieve SOTA results is for the Biosec benchmark. However, in this case, it is possible to achieve a 98.93\% against 100.0\% accuracy of the literature.
These results support our hypothesis that the conception of data-driven systems using deep representations able to extract semantic and vision meaningful features directly from the data is a promising venue. Another indication of this comes from the initial study we did for understanding the type of filters generated by the learning process. Considering the fingerprint case, learning directly from data, it was possible to come up with discriminative filters that explore the blurring artifacts due to recapture. This is particularly interesting as it is in line with previous studies using custom-tailored solutions~\cite{Galbally:TIP:2014}.

It is important to emphasise the interplay between the architecture and filter optimization approaches for the spoofing problem.
It is well-known in the deep learning literature that when thousands of samples are available for learning, the filter learning approach is a promising path. Indeed, we could corroborate this through fingerprint benchmarks that considers a few thousand samples for training. However, it was not the case for faces and two iris benchmarks which suffer from the small sample size problem (SSS) and subject variability hindering the filter learning process. In these cases, the architecture optimization approach was able to learn representative and discriminative features providing comparable spoofing effectiveness to the SOTA results in almost all benchmarks, and specially outperforming them in three out of four SOTA results when the filter learning approach failed. It is worth mentioning that sometimes it is still possible to learn meaningful features from the data even with a small sample size for training. We believe this happens in more well-posed datasets with less variability between training/testing data as it is the case of MobioBIOfake benchmark in which the AO approach achieved 99.38\% just 0.37\% behind the SOTA result. 

As the data tell it all, the decision to which path to follow can also come from the data. Using the evaluation/validation set during training, the researcher/developer can opt for optimizing architectures, learn filters or both. If training time is an issue and a solution must be presented overnight, it might be interesting to consider an already learned network that incorporates some additional knowledge in its design. In this sense, \emph{spoofnet} could be a good choice. In all cases, if the developer can incorporate more training examples, the approaches might benefit from such augmented training data. 

%Although it is more computationally expensive than some existing custom-tailored solutions, our approach is robust to learn representations directly from the data and this is a powerful way of dealing with big data.
%The more data we can collect over time across different modalities, the better and more discriminative will be the representation learning and, consequently, the decision-making process. 
%Moreover, the very fact that the representations are learned directly from the data implies that there is no need for a specialist in each modality or databases for achieving promising results or even outperforming the state-of-the-art methods. 
%If specialist knowledge is available, it can be combined with the proposed solution to provide better-quality training samples and push the solution even further. 

The proposed approaches can also be adapted to other biometric modalities not directly dealt with herein. The most important difference would be in the input type of data since all discussed solutions directly learn their representations from the data.

For the case of iris spoofing detection, here we dealt only with iris spoofing printed attacks and some experimental datasets using cosmetic contact lenses have recently become available allowing researchers to study this specific type of spoofing~\cite{Bowyer:Computer:2014,Yadav:TIFS:2014}. 
For future work, we intend to evaluate such datasets using the proposed approaches here and also consider other biometric modalities such as palm, vein, and gait.

Finally, it is important to take all the results discussed herein with a grain of salt. We are not presenting the final word in spoofing detection. In fact, there are important additional research that could finally take this research another step forward. We envision the application of deep learning representations on top of pre-processed image feature maps (e.g., LBP-like feature maps, acquisition-based maps exploring noise signatures, visual rhythm representations, etc.). With an $n$-layer feature representation, we might be able to explore features otherwise not possible using the raw data. In addition, exploring temporal coherence and fusion would be also important for video-based attacks.

\appendix
\section{ANOVA} 
\label{sec:appendix:anova}

Table~\ref{table:ANOVAT} shows the results of the analysis of variance considering all factors that can influence the value of the AUC for our method (See Table~\ref{table:DOE}). In variance analysis, the assessment of the evidence that the response variable is influenced by the factor {Tr} can be investigated by considering the hypothesis testing shown in Equation~\ref{eq:ht}. If the level means $\alpha_{k}$ of the factor {Tr} are all equal to $0$, then it can be shown that the expected value of the response variable does not depend on factor {Tr}. If at least one level mean of the factor {Tr} is nonzero, then the expected value of the response value of the response variable does depend upon the level of factor {Tr} employed~\cite{Hayter:CL:2012}.
%\newtodo{Allan, use um outro fator, não A pois confunde com o apêndice A.}
%\todo{A equação abaixo deveria ser alinhada à esquerda para $H_i$. Além disso, a hipótese alternativa é classicamente chamada de $H_1$ e não $H_A$. \textbf{Allan:} Feito. \textbf{William}: faltou mudar no texto para 1 e não A.}
\begin{equation}
\begin{aligned}
	H_{0}: & \quad \alpha_{i} = 0 \qquad 1 \leq i \leq k \\
	H_{1}: & \quad \text{some } \alpha_{i} \neq 0
\end{aligned}
\label{eq:ht}
\end{equation}

In the Table~\ref{table:ANOVAT}, the first column shows the factors under analysis and in the second column their respective degree of freedom, which is determined by the number of levels minus one. The ANOVA is based on a decomposition of the total sum of square (SST), into sums of squares for each factor (SSTr) and error sum of squares (SSE), third column of the table, and these measures are important because the plausibility of the null hypothesis that the ``factor level means are all equal'' depends upon the relative size of the sum of squares for treatments SSTr to the sum of squares for error SSE~\cite{Hayter:CL:2012}. With this, we calculate the mean of squares for factors (MSTr) and the mean square error (MSE), shown in the fourth column, by dividing the sum of squares of the each factor by their respective degree of freedom.

Both MSTr and MSE play an important role to determine about the plausibility of the null hypothesis. Consider an hypothetical factor {Tr} with k levels ($1,2, \dots, \alpha_{k}$). If the factor level means $\alpha_{k}$ are all equal, then MSTr and MSE has a $\chi^{2}$ distribution, that is, MSTr $\sim \chi^{2}_{k-1}$ and MSE $\sim \chi^{2}_{n_{T}-k}$. As the F-statistic is calculated by dividing MSTr by MSE, then when the null hypothesis is true, the F-statistic has an F distribution, that is, $F=\frac{MSTr}{MSE} \sim \frac{\chi^{2}_{k-1}}{\chi^{2}_{n_{T}-k}} \sim F_{k-1,n_{T}-k}$.

The plausibility of the null hypothesis is doubtful whenever the observed value of the F-statistic does not look like it is an observation from an $F_{k-1,n_{T}-k}$ distribution. 
%\todo{Não entendi a próxima sentença. \textbf{Allan:} Explicação melhorada} 
The $p$-value showed in the last column of the ANOVA table is a probability of obtain the test statistic ($F_{k-1,n_{T}-k}$) as or more extreme than those observed (F-statistic), assuming that $H_{0}$ is true~\cite{Bland:OG:2002}. Mathematically, $p$-value is calculated as $P(X \geq F)$, which the random variable $X$ has a $F_{k-1,n_{T}-k}$ distribution. A $p$-value smaller than a significance level $\alpha$, leading to rejection of the null hypothesis. Most researchers use a $95\%$ confidence level, that is, $\alpha=5\%$. Therefore, considering the $95\%$ confidence level, we can see that all factors influence significantly the AUC value.
%
\begin{table}[!htb]
\centering
%\linethickness{1.5mm}
\caption{Analysis of variance for the development set of the Replay-Attack database. From left to right column, we have the parameter name, degree of freedom, sum of squares, mean of squares, F-statistic, and $p$-value. Considering a confidence level of $95\%$, the parameters with $p$-values smaller than $0.05$ are important (i.e., significant) to our method.}
\label{table:ANOVAT}
\begin{tabular}{lrrrrr}
\toprule
\textbf{Factor} & \textbf{Df} & \textbf{SS} & \textbf{MS} & \textbf{F-statistic} & \textbf{$\textit{p}$-value} \\
\otoprule
NTV           & 2 &   52601&   26300&  105.93& $<$ 2.2e-16\\
LGF           & 1 &  309848&  309848& 1247.95& $<$ 2.2e-16\\
M             & 3 &   59803&   19934&   80.29& $<$ 2.2e-16\\
CS            & 1 &  160239&  160239&  645.38& $<$ 2.2e-16\\
SDD           & 1 &  492765&  492765& 1984.68& $<$ 2.2e-16\\
DS            & 6 &   36589&    6098&   24.56& $<$ 2.2e-16\\
CP            & 2 &   85974&   42987&  173.14& $<$ 2.2e-16\\
C             & 1 &   16374&   16374&   65.95&  5.08e-16\\
Errors(Residuals)&12078& 2998787&     248 & & \\
\bottomrule
\end{tabular}
\end{table}


\section*{Acknowledgment}

We thank UFOP, Brazilian National Research Counsil -- CNPq (Grants \#303673/2010-9,  \#304352/2012-8, \#307113/2012-4, \#477662/2013-7, \#487529/2013-8, \#479070/2013-0, and \#477457/2013-4), the CAPES DeepEyes project, S\~ao Paulo Research Foundation -- FAPESP, (Grants \#2010/05647-4, \#2011/22749-8, \#2013/04172-0, and \#2013/11359-0), and Minas Gerais Research Foundation -- FAPEMIG (Grant APQ-01806-13). 
D. Menotti thanks FAPESP for a grant to acquiring two NVIDIA GeForce GTX Titan Black with 6GB each.
We also thank NVIDIA for donating five GPUs used in the experiments, a Tesla K40 with 12GB to A. X. Falc{\~a}o, two GeForce GTX 680 with 2GB each to G. Chiachia, and two GeForce GTX Titan Black with 6GB each to D. Menotti.

\ifCLASSOPTIONcaptionsoff
  \newpage
\fi

%\IEEEtriggeratref{8}
%\IEEEtriggercmd{\enlargethispage{-5in}}

%\balance
\bibliographystyle{IEEEtran}
\bibliography{spoofing,iris,faces,fingerprint}
%IEEEabrv,


\begin{IEEEbiography}[{\includegraphics[width=1in,height=1.25in,clip,keepaspectratio]{photo-menotti}}]{David Menotti} received his Computer Engineering and Informatics Applied Master degrees from the Pontif\'icia Universidade Cat\'olica do Paran\'a (PUCPR), Curitiba, Brazil, in 2001 and 2003, respectively.
In 2008, he received his co-tutelage PhD degree in Computer Science from the Federal University of Minas Gerais (UFMG), Belo Horizonte, Brazil and the Universit\'e Paris-Est/Groupe ESIEE, Paris, France.
He is an associate professor at the Computing Department (DECOM), Universidade Federal de Ouro Preto (UFOP), Ouro Preto, Brazil, since August 2008. 
From June 2013 to May 2014 he has been on his sabbatical year at Institute of Computing, University of Campinas (UNICAMP), Campinas, Brazil, as a post-doc / collaborator researcher supported by FAPESP (grant 2013/4172-0).
Currently, he is working as a permanent and collaborator professor at the Post-Graduate Program in Computer Science DECOM-UFOP and DCC-UFMG, respectively.

His research interests include image processing, pattern recognition, computer vision, and information retrieval.
\end{IEEEbiography}

%\vspace{-18pt}
\begin{IEEEbiography}[{\includegraphics[width=1in,height=1.25in,clip,keepaspectratio]{photo-chiachia}}]{Giovani Chiachia}  
Giovani Chiachia is a post-doctorate researcher at University of Campinas, Brazil.
His main research interest is to approach computer vision and other artificial intelligence tasks with brain-inspired machine learning techniques. 
He received his B.Sc. (2005) and his M.S. (2009) degrees from São Paulo State University and his Ph.D. in Computer Science from the University of Campinas (2013).
He held research scholar positions at the Fraunhofer Institute IIS and at Harvard University as part of his graduate work, and was awarded first place for the performance of his face recognition system in the IEEE Intl. Conf. on Biometrics (2013). 
Beyond his research experience, Dr. Chiachia has a large experience in the IT industry, managing and being part of teams working on projects from a wide range of complexities and scales.
\end{IEEEbiography}

%\vspace{-18pt}
\begin{IEEEbiography}[{\includegraphics[width=1in,height=1.25in,clip,keepaspectratio]{photo-pinto}}]{Allan da Silva Pinto} 
Allan Pinto received the B.Sc. degree in Computer Science from University of São Paulo (USP), Brazil, in 2011, and the M.Sc. degree in Computer Science from University of Campinas (Unicamp), Brazil, in 2013. Currently, he is a Ph.D. Student, also in Computer Science, at Institute of Computing, Unicamp, Brazil. 

His research focuses on the areas of image and video analysis, computer forensics, pattern recognition, and computer vision in general, with a particular interest in spoofing detection in biometric systems.
\end{IEEEbiography}

%\vspace{-18pt}
\begin{IEEEbiography}[{\includegraphics[width=1in,height=1.25in,clip,keepaspectratio]{photo-schwartz}}]{William Robson Schwartz} 
William Robson Schwartz received his Ph.D. degree in Computer Science from University of Maryland, College Park, USA. He received his B.Sc. and M.Sc. degrees in Computer Science from the Federal University of Parana, Curitiba, Brazil.
He is currently a professor in the Department of Computer Science at the Federal University of Minas Gerais, Brazil.

His research interests include computer vision, computer forensics, biometrics and image processing.
\end{IEEEbiography}

%\vspace{-18pt}
\begin{IEEEbiography}[{\includegraphics[width=1in,height=1.25in,clip,keepaspectratio]{photo-pedrini}}]{Helio Pedrini} 
Helio Pedrini received his Ph.D. degree in Electrical and Computer Engineering from Rensselaer Polytechnic Institute, Troy, NY, USA. 
He received his M.Sc. degree in Electrical Engineering and his B.Sc. in Computer Science, both from the University of Campinas, Brazil. 
He is currently a professor in the Institute of Computing at the University of Campinas, Brazil. 

His research interests include image processing, computer vision, pattern recognition, machine learning, computer graphics, computational geometry.
\end{IEEEbiography}

%\vspace{-18pt}
\begin{IEEEbiography}[{\includegraphics[width=1in,height=1.25in,clip,keepaspectratio]{photo-falcao}}]{Alexandre Xavier Falcão} 
Alexandre Xavier Falcao is full professor at the Institute of Computing, University of Campinas, Campinas, SP, Brazil. 
He received a B.Sc. in Electrical Engineering from the Federal University of Pernambuco, Recife, PE, Brazil, in 1988. He has worked in biomedical image processing, visualization and analysis since 1991. 
In 1993, he received a M.Sc. in Electrical Engineering from the University of Campinas, Campinas, SP, Brazil. 
During 1994-1996, he worked with the Medical Image Processing Group at the Department of Radiology, University of Pennsylvania, PA, USA, on interactive image segmentation for his doctorate. 
He got his doctorate in Electrical Engineering from the University of Campinas in 1996. 
In 1997, he worked in a project for Globo TV at a research center, CPqD-TELEBRAS in Campinas, developing methods for video quality assessment. 
His experience as professor of Computer Science and Engineering started in 1998 at the University of Campinas. 

His main research interests include image/video processing, visualization, and analysis; graph algorithms and dynamic programming; image annotation, organization, and retrieval; machine learning and pattern recognition; and image analysis applications in Biology, Medicine, Biometrics, Geology, and Agriculture.
\end{IEEEbiography}

%\vspace{-18pt}
\begin{IEEEbiography}[{\includegraphics[width=1in,height=1.25in,clip,keepaspectratio]{photo-rocha}}]{Alexandre de Rezende Rocha}
Anderson de Rezende Rocha received his B.Sc (Computer Science) degree from Federal University of Lavras (UFLA), Brazil in 2003. He received his M.S. and Ph.D. (Computer Science) from University of Campinas (Unicamp), Brazil, in 2006 and 2009, respectively. 
Currently, he is an assistant professor in the Institute of Computing, Unicamp, Brazil. 

His main interests include digital forensics, reasoning for complex data, and machine  intelligence. 
He has actively worked as a program committee member in several important Computer Vision, Pattern Recognition, and Digital Forensics events. 
He is an associate editor of the Elsevier Journal of Visual Communication and Image Representation (JVCI) and a Leading Guest Editor for EURASIP/Springer Journal on Image and Video Processing 
(JIVP) and IEEE Transactions on Information Forensics and Security (T.IFS). 
He is an affiliate member of the Brazilian Academy of Sciences (ABC) and the Brazilian Academy of Forensics Sciences (ABCF). 
He is also a Microsoft Research Faculty Fellow and a former member of the IEEE Information Forensics and Security Technical Committee (IFS-TC).
\end{IEEEbiography}
%\balance

\end{document}
